
\section{Introducción}
\label{sec:Intr}
La diabetes mellitus es reconocida mundialmente como uno de los principales desafíos sanitarios del siglo XXI, debido a su elevada prevalencia y a las múltiples complicaciones asociadas que afectan significativamente la calidad de vida de los pacientes, así como los costos para los sistemas de salud \cite{ministerio2022estrategia}. Esta enfermedad crónica no solo genera un impacto directo sobre quienes la padecen, sino también sobre sus familias y comunidades, debido al manejo continuo y a largo plazo que exige. Las complicaciones derivadas de la diabetes, tales como enfermedades cardiovasculares, nefropatías y neuropatías, incrementan significativamente la morbilidad y disminuyen la esperanza de vida de los pacientes.

Entre estas complicaciones, las heridas ulcerosas en pacientes con pie diabético destacan por su alta incidencia, gravedad clínica y complejidad en el manejo terapéutico. Dichas heridas representan una carga considerable para los sistemas de salud debido a su tendencia a la cronicidad, las frecuentes recaídas y el alto riesgo de infecciones graves que pueden culminar en amputaciones subsecuentes \cite{mishra2017diabetic,bandyk2018diabetic}. De hecho, las heridas en pie diabético constituyen una de las principales causas de hospitalización prolongada y discapacidad permanente en estos pacientes.

Actualmente, la evaluación clínica de las heridas ulcerosas depende predominantemente de criterios visuales subjetivos, generando inconsistencias diagnósticas y retrasos terapéuticos. Esta situación se agudiza con el aumento sostenido de la diabetes mellitus registrado por la Encuesta Nacional de Salud, que pasó de 6,4% en 2003 a 12,3% en 2016-2017. A su vez, la coexistencia de múltiples patologías crónicas en los pacientes incrementa la complejidad asistencial y demanda enfoques integrales basados en evidencia.

En respuesta a este contexto, surge la necesidad de desarrollar herramientas objetivas y cuantitativas para mejorar el seguimiento y la evaluación del proceso de cicatrización en heridas ulcerosas del pie diabético. Esta investigación y desarrollo propone enfrentar dicha necesidad mediante el desarrollo de una herramienta automatizada que estime objetivamente el estadio de estas heridas a partir de técnicas avanzadas de procesamiento de imágenes y aprendizaje automático. La herramienta propuesta integrará métodos de segmentación automática de heridas utilizando imágenes capturadas mediante dispositivos móviles, la extracción y análisis de características radiómicas empleando técnicas especializadas como pyradiomics \cite{van2017computational}, y algoritmos avanzados de machine learning para una estimación precisa del estadio clínico según la escala PWAT (Photographic Wound Assessment Tool) \cite{thompson2013reliability}. Además, se contemplará la incorporación de información clínica relevante del paciente mediante registros digitales, buscando así una evaluación integral que permita una gestión clínica más eficiente.

De esta manera, la implementación de esta herramienta busca mejorar significativamente la precisión diagnóstica, reducir complicaciones mayores, favorecer intervenciones oportunas y eficientes, y optimizar el uso de recursos sanitarios disponibles, respondiendo así efectivamente a un desafío sanitario de creciente relevancia global \cite{organizacion2016informe}.
El resto de este documento se organiza de la siguiente manera: el Capítulo~\ref{ch:MC} aborda el marco conceptual y el estado del arte; el Capítulo~\ref{ch:Problema} describe el problema y la solución propuesta; el Capítulo~\ref{ch:OG} presenta los objetivos; el Capítulo~\ref{ch:Met} expone la metodología junto con la planificación; el Capítulo~\ref{ch:req} detalla los requerimientos; el Capítulo~\ref{ch:Impl} profundiza en el diseño de la plataforma; y el Capítulo~\ref{ch:rec} resume los recursos necesarios.
