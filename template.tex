\documentclass[12pt,letterpaper]{report}

\marginparsep 0pt
\textwidth 6in
\topmargin 0pt
\headsep .5in
\textheight 9.2in
\voffset = 0pt
\hoffset = 0pt
\marginparwidth = 0pt \oddsidemargin = 0pt \sloppy

%Dimensiones de la página
\usepackage[left=2.5cm,top=3cm,right=2.5cm,bottom=2.5cm]{geometry}
%Sangría
\setlength{\parindent}{1cm}

%Numeracion
\pagenumbering{arabic}
\usepackage{graphicx}    % ya lo tendrás cargado
\usepackage{adjustbox}   % añade \adjustbox{}
\usepackage{tikz}
\usepackage{template}
\usepackage{amsmath,amsfonts}
\usepackage{graphicx}
\usepackage{graphics}
\usepackage[dvips]{epsfig}
\usepackage{times}
\usepackage[utf8]{inputenc}
\usepackage[dvips]{graphicx}
\usepackage[dvipsnames]{xcolor}
\usepackage[spanish]{babel}
\newcommand{\ie}{i.e.}
\newcommand {\out}[1]{}
\newtheorem{definicion}{Definicion}
\usepackage[dvips]{epsfig}
\usepackage{rotating}
\usepackage{multirow}
\usepackage{array}
\usepackage{longtable}
\usepackage[]{fontenc}
\usepackage{hyperref}
\usepackage{float}
\usepackage{tabularx}
\usepackage{booktabs}
\renewcommand{\shorthandsspanish}{}

\addto\captionsspanish{
\def\listtablename{Índice de tablas}
\def\tablename{Tabla}}

\sloppy

\begin{document}
\title{\textbf{Herramienta de apoyo automática en la estimación del estadio de heridas ulcerosas en pacientes con pie diabético}}
\author{\textbf{Benjamin Alejandro Morales Carvajal}}
\principaladviser{Ana Aguilera Faraco}
\coprincipaladviser{Julio Sotelo Parraguez, Marcelo Marquez}


\beforepreface
\prefacesection{Resumen}

Las heridas ulcerosas en pacientes con pie diabético representan una complicación clínica de alta prevalencia y riesgo, que afecta la calidad de vida de los pacientes y genera elevados costos sanitarios. Tradicionalmente, su evaluación depende de criterios visuales subjetivos, lo que provoca variabilidad diagnóstica y retrasos en las decisiones terapéuticas; aunque se utiliza la escala PWAT (Photographic Wound Assessment Tool), esta aún presenta limitaciones en objetividad y automatización.

Para abordar esta problemática, se desarrollará una herramienta automatizada que integra un modelo avanzado de segmentación, extracción de características radiómicas mediante pyradiomics y algoritmos de machine learning, complementados con información clínica adicional para estimar objetivamente el puntaje PWAT.

Los resultados demostraron una mejora significativa en precisión y reproducibilidad frente a métodos manuales, posibilitando una evaluación integral del paciente y promoviendo intervenciones clínicas más oportunas y eficientes. De este modo, la metodología aportó robustez y escalabilidad al estado del arte en la valoración de heridas ulcerosas.

En un contexto más amplio, la herramienta contribuirá a reducir complicaciones graves, optimizar la asignación de recursos sanitarios y mejorar la atención sanitaria general.
%A continuación aspectos que debe considera para escribir un resumen comprensible y que cumpla su %propósito.
%
%\begin{itemize}
%    \item Una o dos oraciones que provean una introducción básica al área de trabajo, comprensibles %para un interesado de cualquier disciplina.
%    \item Dos o tres oraciones de antecedentes más detallados, comprensibles para personas de %disciplinas relacionadas.
%    \item Una oración que aclara el problema general que trata en este trabajo de título.
%    \item Una oración que resume el resultado principal.
%    \item Dos o tres oraciones que expliquen el aporte de sus resultados al estado del arte.
%    \item Una o dos oraciones para situar los resultados en un contexto más general.
%    \item (opcional) Dos o tres oraciones para proporcionar una perspectiva más amplia, fácilmente %comprensible para una persona de cualquier disciplina.
%
%\end{itemize}
%Recuerde 
%\begin{itemize}
%    \item El resumen es una consolidación de su trabajo.
%    \item Debe ser conciso y describir el alcance de su trabajo, resumir sus resultados y presentar %las principales conclusiones. 
%    \item Debe tener máximo 250 palabras. 
%    \item Debe ser escrito en pasado. 
%    
%\end{itemize}
%
%\textit{Importante: Tome los elementos que sean pertinentes a la etapa en que está de su trabajo de título. Estos aspectos están pensados para su documento final.}
%\newpage
%\prefacesection{Agradecimientos}
%Aqui pueden colocar sus agradecimientos.
%Si han estudiado con becas es recomendable colocar los agradecimientos a las instituciones que les otorgaron las becas.

%\afterpreface

\tableofcontents
\newpage

%%=============================================================================
%% List of Figures

%\renewcommand{\listfigurename}{List of Figures}
\listoffigures
\newpage

%%=============================================================================
%% List of Tables
\renewcommand{\listtablename}{Índice de tablas} 
\renewcommand{\tablename}{Tabla} 

\listoftables
\newpage

\pagenumbering{arabic}

%Aquí deben incluir el fuente de cada capítulo, sin su encabezado:

\chapter{Introducci\'on}
\label{ch:Intro}

\section{Introducción}
\label{sec:Intr}
La diabetes mellitus es reconocida mundialmente como uno de los principales desafíos sanitarios del siglo XXI, debido a su elevada prevalencia y a las múltiples complicaciones asociadas que afectan significativamente la calidad de vida de los pacientes, así como los costos para los sistemas de salud \cite{ministerio2022estrategia}. Esta enfermedad crónica no solo genera un impacto directo sobre quienes la padecen, sino también sobre sus familias y comunidades, debido al manejo continuo y a largo plazo que exige. Las complicaciones derivadas de la diabetes, tales como enfermedades cardiovasculares, nefropatías y neuropatías, incrementan significativamente la morbilidad y disminuyen la esperanza de vida de los pacientes.

Entre estas complicaciones, las heridas ulcerosas en pacientes con pie diabético destacan por su alta incidencia, gravedad clínica y complejidad en el manejo terapéutico. Dichas heridas representan una carga considerable para los sistemas de salud debido a su tendencia a la cronicidad, las frecuentes recaídas y el alto riesgo de infecciones graves que pueden culminar en amputaciones subsecuentes \cite{mishra2017diabetic,bandyk2018diabetic}. De hecho, las heridas en pie diabético constituyen una de las principales causas de hospitalización prolongada y discapacidad permanente en estos pacientes.

Actualmente, la evaluación clínica de las heridas ulcerosas depende predominantemente de criterios visuales subjetivos, generando inconsistencias diagnósticas y retrasos terapéuticos. Esta situación se agudiza con el aumento sostenido de la diabetes mellitus registrado por la Encuesta Nacional de Salud, que pasó de 6,4% en 2003 a 12,3% en 2016-2017. A su vez, la coexistencia de múltiples patologías crónicas en los pacientes incrementa la complejidad asistencial y demanda enfoques integrales basados en evidencia.

En respuesta a este contexto, surge la necesidad de desarrollar herramientas objetivas y cuantitativas para mejorar el seguimiento y la evaluación del proceso de cicatrización en heridas ulcerosas del pie diabético. Esta investigación y desarrollo propone enfrentar dicha necesidad mediante el desarrollo de una herramienta automatizada que estime objetivamente el estadio de estas heridas a partir de técnicas avanzadas de procesamiento de imágenes y aprendizaje automático. La herramienta propuesta integrará métodos de segmentación automática de heridas utilizando imágenes capturadas mediante dispositivos móviles, la extracción y análisis de características radiómicas empleando técnicas especializadas como pyradiomics \cite{van2017computational}, y algoritmos avanzados de machine learning para una estimación precisa del estadio clínico según la escala PWAT (Photographic Wound Assessment Tool) \cite{thompson2013reliability}. Además, se contemplará la incorporación de información clínica relevante del paciente mediante registros digitales, buscando así una evaluación integral que permita una gestión clínica más eficiente.

De esta manera, la implementación de esta herramienta busca mejorar significativamente la precisión diagnóstica, reducir complicaciones mayores, favorecer intervenciones oportunas y eficientes, y optimizar el uso de recursos sanitarios disponibles, respondiendo así efectivamente a un desafío sanitario de creciente relevancia global \cite{organizacion2016informe}.
El resto de este documento se organiza de la siguiente manera: el Capítulo~\ref{ch:MC} aborda el marco conceptual y el estado del arte; el Capítulo~\ref{ch:Problema} describe el problema y la solución propuesta; el Capítulo~\ref{ch:OG} presenta los objetivos; el Capítulo~\ref{ch:Met} expone la metodología junto con la planificación; el Capítulo~\ref{ch:req} detalla los requerimientos; el Capítulo~\ref{ch:Impl} profundiza en el diseño de la plataforma; y el Capítulo~\ref{ch:rec} resume los recursos necesarios.


\chapter{Marco Conceptual y Estado del Arte}
\label{ch:MC}
                

\section{Marco Conceptual}
\label{sc:MC}


\subsection{Definición y clasificación de las heridas crónicas en pie diabético}
Las heridas crónicas son lesiones que, a diferencia de las heridas agudas, no logran completar un proceso normal de cicatrización en los tiempos esperados. Típicamente, si una herida no muestra signos claros de cierre tras $\sim 4$ semanas, se considera crónica o de difícil cicatrización \cite{healthcare11020273}. Estas lesiones constituyen un serio problema de salud publica a nivel global, afectando a millones de personas y generando costos sanitarios elevados (por ejemplo, se estiman mas de 25 mil millones de dólares anuales solo en Estados Unidos) \cite{healthcare11020273}. En pacientes con diabetes mellitus, las \textbf{ulceras de pie diabético} destacan como una de las complicaciones crónicas mas frecuentes y graves; presentan alta incidencia, manejo terapéutico complejo, y tienden a la cronicidad con recidivas frecuentes. Ademas, implican un alto riesgo de infecciones severas que pueden culminar en amputaciones; de hecho, las ulceras de pie diabético son una de las principales causas de hospitalización prolongada y discapacidad permanente en personas diabeticas. La diabetes y otras comorbilidades (ej. obesidad, arteriopatías) contribuyen a interrumpir el proceso normal de cicatrización, por lo que estos pacientes son propensos a desarrollar heridas crónicas \cite{healthcare11020273}.
Desde el punto de vista clínico, las heridas crónicas se clasifican según su etiología y gravedad. Se incluyen, entre otras, las úlceras por presión, úlceras venosas y las úlceras neuropáticas o isquémicas del diabético. En el caso específico del pie diabético, existen sistemas de estadificación para graduar la severidad de la úlcera; por ejemplo, la clasificación de Wagner (grados 0 a 5) se emplea para describir la profundidad y extensión del daño tisular, desde lesiones superficiales limitadas a la epidermis (grado 1) hasta gangrena extensa que compromete todo el pie (grado 5) \cite{healthcare11020273}. Estas clasificaciones ayudan a orientar el manejo terapéutico y pronóstico de la herida. Sin embargo, cabe señalar que muchas de estas escalas clínicas (Wagner y otras) dependen de la apreciación del evaluador, introduciendo cierto grado de subjetividad en la determinación del estadio de la lesión. En resumen, las heridas crónicas –y en particular las úlceras diabéticas– representan un desafío sanitario por su alta prevalencia y severidad, requiriendo sistemas estandarizados para su clasificación y seguimiento.

\subsection{Proceso Fisiológico de Cicatrización}

La cicatrización normal de una herida ocurre a través de una serie de fases fisiológicas solapadas, durante las cuales el tejido lesionado es reemplazado por tejido reparado. En términos generales, se distinguen tres etapas principales \cite{ulcerasCicatrizacixF3nxDAlcerasnet}:
\begin{enumerate}
    \item \textbf{Fase inflamatoria}
    \begin{enumerate}
        \item Ocurre inmediatamente tras la lesión y se extiende por los primeros días (aprox. días 0 al 4). En esta etapa se detiene la hemorragia mediante vasoconstricción y formación de coágulo (hemostasia) y se desencadena la respuesta inflamatoria local. Llegan neutrófilos y macrófagos al lecho de la herida para eliminar bacterias y tejido dañado, limpiando la zona y liberando citocinas y factores de crecimiento que inician la reparación \cite{ulcerasCicatrizacixF3nxDAlcerasnet}. Esta fase prepara el terreno para la regeneración, pero en heridas complicadas puede prolongarse más de lo normal, especialmente si persisten factores irritantes o infección crónica.
    \end{enumerate}
    \item \textbf{Fase Proliferativa}
    \begin{enumerate}
        \item Es la etapa intermedia, típicamente abarcando desde  $\sim$ el día 4 hasta la segunda semana post-lesión.Durante esta fase se forma tejido de granulación que llena el defecto de la herida: prolifera nueva vasculatura (angiogénesis) y fibroblastos que depositan matriz extracelular y colágeno. A la par, ocurre la reepitelización, con migración y división de queratinocitos desde los bordes (y anexos cutáneos remanentes) para cubrir la superficie lesionada. También se inicia la contracción de la herida gracias a miofibroblastos, acercando los bordes para reducir el área abierta \cite{ulcerasCicatrizacixF3nxDAlcerasnet}. Esta fase proliferativa es crucial para cerrar la herida; su eficacia determina en gran medida el pronóstico de cicatrización.
    \end{enumerate}
    \item \textbf{Fase de maduración (remodelación)}
    \begin{enumerate}
        \item Es la fase final, que puede prolongarse por meses luego de cerrada la herida. En esta etapa el nuevo tejido conectivo se reorganiza y refuerza: las fibras de colágeno tipo III depositadas en granulación se remodelan a colágeno tipo I más resistente, aumentando gradualmente la fuerza tensil de la cicatriz. Los vasos sanguíneos neoformados en exceso involucionan y la inflamación residual desaparece, quedando una cicatriz más pálida y firme. Aunque la herida luzca cerrada, internamente continúa este remodelado que puede durar de 6 a 12 meses (dependiendo del tamaño y profundidad de la lesión).
    \end{enumerate}
\end{enumerate}


\subsection{Evaluación Clínica de las heridas y escalas existentes}

La evaluación clínica tradicional de las heridas se basa predominantemente en la inspección visual y mediciones manuales, lo cual introduce un componente subjetivo significativo. Por ejemplo, el método clásico para medir el tamaño de una úlcera es usando una regla para longitud/ancho y una sonda o hisopo para estimar la profundidad; sin embargo, este método es laborioso, poco preciso para áreas irregulares y conlleva riesgo de introducir infección al contacto con la herida \cite{healthcare11020273}. Adicionalmente, la apreciación de aspectos cualitativos –como el tipo de tejido presente o el grado de humedad/exudado– depende del criterio del profesional. Esto genera variabilidad \textbf{inter e intra-observador:} distintos clínicos pueden estimar porcentajes de tejido de granulación vs. necrótico de forma diferente, llevando a inconsistencias diagnósticas y diferencias en decisiones terapéuticas. De hecho, la literatura reporta que la evaluación visual de heridas ulcerosas es altamente subjetiva, pudiendo retrasar intervenciones óptimas si se subestima la severidad de la lesión.

Para mejorar la objetividad y estandarización, se han desarrollado herramientas y escalas de evaluación de heridas. Una de ellas es la \textbf{Bates-Jensen Wound Assessment Tool (BWAT)} \cite{harris2010bates}, anteriormente conocida como escala PSST, que evalúa clínicamente múltiples parámetros de la herida (tamaño, profundidad, bordes, tipo de tejido, exudado, etc.) asignando puntuaciones que reflejan el estado de cicatrización. Otra escala común es la \textbf{Pressure Ulcer Scale for Healing (PUSH)} \cite{stotts2001instrument}, enfocada en úlceras por presión, que monitoriza la evolución mediante un puntaje derivado de área, cantidad de exudado y tipo de tejido predominante. En el contexto de la documentación fotográfica, destaca el \textbf{Photographic Wound Assessment Tool (PWAT)} \cite{Curti2024}, una herramienta específica para evaluar heridas a partir de fotografías estandarizadas. El PWAT cuantifica ocho dominios (tamaño de la herida, profundidad/compromiso de tejidos, tipo de tejido necrótico, cantidad de tejido necrótico, tipo de tejido de granulación, cantidad de granulación, condición de los bordes y condición de la piel perilesional) otorgando a cada uno un sub-puntaje; la suma total varía de 0 (herida cerrada) a 32 (herida de peor aspecto). Esta escala fotográfica permite un seguimiento objetivo de la cicatrización: una disminución del puntaje PWAT en sucesivas evaluaciones indica que la herida está sanando. En teoría, el PWAT reduce la subjetividad al ofrecer descripciones precisas para cada categoría de puntaje (por ejemplo, define criterios claros para “tipo de tejido necrótico” o “viabilidad de la piel periférica”).

Pese a la existencia de estas escalas estandarizadas, su implementación práctica presenta limitaciones. Aplicar herramientas como BWAT o PWAT requiere tiempo adicional en clínica para realizar la puntuación, y cierta capacitación del personal para usarlas consistentemente. En entornos de alta demanda asistencial, a veces se omite el uso sistemático de estas escalas, manteniéndose evaluaciones más bien cualitativas. Además, no eliminan por completo la subjetividad –por ejemplo, dos profesionales podrían discrepar ligeramente en la interpretación de “\% de tejido necrótico” en una foto al usar PWAT, aunque la escala delimite rangos. De hecho, incluso con apoyo de imágenes, sigue habiendo variabilidad entre evaluadores humanos. Otra barrera es la actualización y registro de estos puntajes en historias clínicas, que puede ser percibido como trabajo administrativo extra.

En la práctica clínica diaria, \textbf{la clasificación del pie diabético} a menudo combina herramientas: se usa la clasificación de Wagner para grado de la úlcera, y escalas de riesgo de úlceras por presión (Braden, Norton) si el paciente está encamado, junto con inspecciones seriadas de la herida para evaluar signos de mejoría o deterioro (disminución de tamaño, aumento de tejido granulado, etc.). No obstante, dada la subjetividad inherente, se reconoce la necesidad de \textbf{métodos más objetivos y automáticos} para evaluar las heridas ulcerosas de forma más confiable. En años recientes se han introducido dispositivos y aplicaciones móviles que ayudan a medir digitalmente las heridas: por ejemplo, sistemas de fotografía con calibración que calculan el área de la lesión a partir de la imagen
. Algunos de estos incluso diferencian áreas de tejido dentro de la herida, aunque pueden requerir que el usuario marque manualmente puntos de referencia o delimite aproximadamente la zona de interés para cada tejido
. Si bien estas tecnologías han aliviado algunas limitaciones (p.ej. eliminan el contacto físico con la herida durante la medición
), \textbf{aún no están ampliamente difundidas}, y muchas dependen de hardware específico o condiciones controladas. Esto sienta el escenario para la incursión de técnicas de inteligencia artificial, que prometen automatizar completamente la evaluación de heridas a partir de imágenes, reduciendo la dependencia en la subjetividad humana.

\subsection{Aplicaciones de inteligencia artificial en el análisis de imágenes de heridas}

En la última década, la inteligencia artificial (IA) –especialmente las técnicas de deep learning o aprendizaje profundo– ha irrumpido en el campo del análisis de imágenes médicas, incluyendo el cuidado de heridas. El objetivo es asistir o automatizar tareas como la medición del tamaño de la herida, la segmentación de los distintos tejidos en la úlcera y la estimación del estadio de cicatrización, todo a partir de fotografías digitales. Diversos estudios recientes han explorado enfoques de IA para superar las deficiencias del método manual de la regla, con resultados prometedores
. De hecho, Chairat  (2023) señalan que la IA tiene el potencial de proporcionar mediciones objetivas y precisas de características clave de la herida (área, proporción de tejidos, etc.), sirviendo como base para una evaluación y planificación de tratamiento más informada
. En esencia, un sistema de IA bien entrenado podría fungir como una “segunda opinión” automatizada, cuantificando la herida de forma consistente en cada control.

\textbf{D. M. Anisuzzaman  (2020)} llevaron a cabo una \textbf{revisión sistemática} exhaustiva sobre el uso de inteligencia artificial basada en imágenes para la evaluación de heridas \cite{https://doi.org/10.48550/arxiv.2009.07141}. En dicha revisión se recopiló y analizó más de un centenar de estudios relevantes, cubriendo tanto métodos de \textbf{medición/segmentación} de heridas como técnicas de \textbf{diagnóstico/clasificación} automatizada, además de sistemas integrales (incluyendo hardware especializado, software y aplicaciones móviles) desarrollados hasta la fecha.

Los hallazgos de Anisuzzaman  muestran que las aproximaciones de IA aplicadas a heridas han evolucionado desde algoritmos relativamente simples (basados en reglas fijas de procesamiento de la imagen) hasta modelos complejos de aprendizaje profundo. Inicialmente, varios trabajos emplearon métodos \textbf{basados en reglas o algoritmos de visión tradicionales}, por ejemplo, segmentación por umbral de color o clustering (agrupamiento) de pixeles para distinguir áreas de la herida.

Posteriormente, se introdujeron técnicas de \textbf{aprendizaje automático clásico}: clasificadores como support vector machines (SVM) o árboles de decisión que, usando características extraídas de la imagen (p.ej., estadísticas de color o textura), intentaban distinguir entre heridas que cicatrizan bien vs. mal, o entre distintos tipos de lesiones \cite{Curti2024}. En años recientes, el auge del \textbf{aprendizaje profundo} ha llevado a que predominen los modelos de \textbf{redes neuronales convolucionales (CNN)} entrenados de extremo a extremo con grandes conjuntos de imágenes de heridas. Estos modelos aprenden directamente de los píxeles a segmentar y clasificar, muchas veces superando el desempeño de las técnicas anteriores al poder captar patrones más complejos

Ejemplo de lo anterior es el trabajo de \textbf{Sawrawit Chairat  (2023)}, quienes desarrollaron un sistema automático de evaluación de heridas mediante algoritmos de \textit{deep learning}. En su estudio, emplearon arquitecturas de segmentación tipo \textbf{U-Net} combinadas con redes pre-entrenadas (EfficientNet-B2 y MobileNetV2) para analizar imágenes de úlceras capturadas con smartphones \cite{healthcare11020273}. El sistema de Chairat  integra además una etapa previa de \textbf{calibración de color y tamaño} usando una carta de calibración física colocada junto a la herida, lo que permite corregir diferencias de iluminación y escala entre fotografías. Los modelos de \textit{deep learning} así entrenados fueron capaces de \textbf{segmentar automáticamente} tanto el contorno del área total de la herida como las regiones correspondientes a tejido de granulación, tejido necrótico y área epitelial en cada imagen. 
Los resultados reportados indican una precisión elevada en algunos componentes: el mejor modelo logró un Índice de Jaccard (IoU) de  $\sim 0,70$ para la segmentación del área de la herida y  $\sim 0,64$ para la del tejido de granulación, lo que implica buena concordancia con lo delineado por expertos humanos \cite{Shi2014}. 
En cambio, el desempeño fue más moderado para la zona de epitelización (IoU  $\sim 0,40$) y bajo para áreas necróticas (IoU  $\sim 0,16$), evidenciando la mayor dificultad de la IA para identificar tejidos minoritarios o poco diferenciados en las imágenes. Aun con esas limitaciones, la herramienta demostró poder replicar de forma consistente las mediciones que usualmente realizan manualmente los clínicos, especialmente en lo referente al tamaño de la herida y al porcentaje de tejido de granulación. Los autores destacaron además que la incorporación de la carta de calibración mejoró el rendimiento del algoritmo, al estandarizar los colores y dimensiones en las fotos analizadas. Esto subraya la importancia de controlar las variables de imagen (iluminación, distancia, etc.) al aplicar IA en entornos clínicos reales.

Otro avance notable es el de \textbf{Curti}, quienes propusieron un sistema automatizado enfocado en \textbf{predecir la puntuación PWAT} de una herida a partir de la imagen. En este enfoque, se combina la segmentación automática con análisis de características (\textit{radiomics}): primero, una red neuronal identifica y delimita la herida en la fotografía; luego, sobre esa región, se extraen múltiples descriptores texturales y morfológicos diseñados para imitar los criterios que tendría un clínico al evaluar la herida (por ejemplo, rasgos que reflejan cantidad de tejido de granulación, presencia de bordes epiteliales, etc.). Finalmente, un modelo de regresión utiliza esas características para \textbf{estimar el puntaje PWAT} que correspondería a la imagen. Este sistema fue entrenado y validado con un amplio conjunto de imágenes anotadas por expertos, logrando resultados muy positivos: la puntuación PWAT predicha por la IA tuvo una correlación de alrededor de r = 0,85 (Spearman) con la puntuación otorgada por dermatólogos en imágenes de prueba no vistas por el modelo. En otras palabras, el algoritmo replicó en gran medida la evaluación humana, lo que proporciona un benchmark alentador para futuras aplicaciones de IA en este campo. Cabe destacar que los autores enfatizan que las características utilizadas por el modelo son fácilmente interpretables por clínicos (tienen un significado médico claro), lo cual facilita la aceptación de este tipo de herramientas como sistemas de apoyo en la toma de decisiones \cite{Curti2024}.

La revisión de \textbf{Anisuzzaman } también resalta desarrollos en sistemas integrales para el cuidado de heridas. Por ejemplo, se han creado aplicaciones móviles que, apoyadas en redes neuronales, permiten al profesional tomar una foto de la úlcera con el teléfono y obtener instantáneamente mediciones de área y sugerencias sobre el tejido predominante \cite{https://doi.org/10.48550/arxiv.2009.07141}. También existen prototipos de dispositivos portátiles específicos para clínicas de heridas, que combinan cámaras especializadas (incluso 3D) con software de análisis para ofrecer un seguimiento cuantitativo de la cicatrización. Muchos de estos están en etapas de evaluación piloto, pero muestran la \textbf{tendencia} clara hacia la digitalización y automatización en la gestión de heridas crónicas. La IA, en conjunto con la disponibilidad cada vez mayor de cámaras de alta calidad en dispositivos móviles, habilita posibilidades como telemedicina en el cuidado de heridas (evaluar remotamente una herida a través de una foto enviada y analizada con algoritmos), o sistemas de alerta temprana que identifiquen si una herida está empeorando para recomendar consulta médica prioritaria. En resumen, las aplicaciones de IA en imágenes de heridas abarcan desde la segmentación precisa y objetiva de la herida hasta la predicción inteligente de índices clínicos de cicatrización, y representan un campo en rápida expansión dentro de la informática médica.

\subsection{Técnicas de segmentación y clasificación de tejidos en la herida}

Un componente esencial para estimar el estado de una herida crónica es \textbf{determinar la composición de su lecho}, es decir, qué proporción corresponde a tejido de granulación viable, cuánto es tejido necrótico o desvitalizado, y cuánto es epitelio nuevo en formación. Estos parámetros ofrecen información directa sobre la fase de cicatrización en la que se encuentra la úlcera: por ejemplo, un aumento del tejido de granulación rojo y brillante junto con la aparición de bordes epiteliales suele indicar progreso hacia la curación, mientras que la presencia predominante de tejido necrótico (fibrina amarilla o escara negra) señala estancamiento y riesgo de infección \cite{ulcerasCicatrizacixF3nxDAlcerasnet}. Por ello, las guías clínicas y herramientas de evaluación de heridas (como BWAT o PWAT) incluyen la estimación visual del porcentaje de cada tipo de tejido como un \textbf{indicador del estado de la herida}. La \textbf{granulación} se identifica típicamente por su aspecto rojizo húmedo (tejido nuevo con capilares abundantes), la \textbf{necrosis} por zonas de color amarillo, gris o negro (tejido muerto que el cuerpo no ha eliminado), y la \textbf{epitelización} por parches rosados claros en los bordes o islas en el interior (piel nueva en formación). Un seguimiento cuidadoso de estos componentes en el tiempo permite inferir si la herida avanza o retrocede en su proceso de reparación. 

La \textbf{segmentación automática} de los tejidos de una herida a partir de imágenes es técnicamente compleja debido a la variabilidad visual de las heridas (colores, texturas) y a factores de la imagen (iluminación, contraste). Los primeros intentos en este campo se basaron en técnicas tradicionales de \textbf{procesamiento digital de imágenes}. Entre ellas se cuentan los algoritmos de \textbf{segmentación por umbral y color}, donde se convertía la imagen a un espacio de color adecuado (por ejemplo CIELab) y se aplicaban umbrales para separar regiones rojas, amarillas, negras, etc. correspondientes a granulación, fibrina o escara. También se exploraron métodos de \textbf{segmentación por superpíxeles y crecimiento de regiones}, en los cuales la imagen se subdivide en pequeñas regiones homogéneas que luego se agrupan según similitud de color/textura para identificar distintas zonas de tejido. Adicionalmente, se diseñaron \textbf{sistemas expertos basados en reglas}, donde mediante conocimiento de especialistas se programaban criterios (p.ej., “si la región es roja brillante con cierta saturación, clasificar como granulación”). Estos enfoques no requieren datos de entrenamiento extensos, apoyándose en heurísticas fijas, y en algunos casos lograron resultados razonables en escenarios controlados. No obstante, enfrentan dificultades para generalizar: las heridas reales presentan gran diversidad y las condiciones de las fotos pueden cambiar (iluminación, presencia de gasas, sangrado, etc.), por lo que las reglas rígidas a menudo fallan fuera de un conjunto limitado de casos. 

Con el advenimiento del \textit{deep learning}, la segmentación de tejidos en heridas ha dado un salto en precisión y robustez. Las \textbf{redes neuronales convolucionales (CNN)} especializadas en segmentación semántica (como \textbf{U-Net, SegNet, DeepLab}, entre otras) se entrenan con conjuntos de imágenes de heridas donde cada pixel está etiquetado según el tipo de superficie que representa (fondo, herida, granulación, necrosis, epitelio, etc.). Estos modelos aprenden características visuales de alto nivel que distinguen automáticamente las regiones de interés. Un enfoque común es utilizar arquitecturas de \textbf{multi-tarea}, en que una misma red puede sacar varios mapas de salida: uno para segmentar la herida completa (frente a la piel intacta) y otro para segmentar dentro de la herida los diferentes tejidos. Esto es posible porque los distintos tipos de tejido en el lecho son excluyentes (una zona de la herida o es granulación, o es necrosis, etc.), y la red puede distribuir su capacidad en reconocer cada categoría. Los estudios indican que las CNN bien entrenadas pueden \textbf{clasificar píxeles de la herida en categorías de tejido con alta velocidad y exactitud}, superando a los algoritmos basados en umbrales en consistencia. Por ejemplo, como se mencionó antes, la aplicación de U-Net por Chairat  logró identificar correctamente la mayor parte de las áreas de granulación en las imágenes de prueba.

Pese a los progresos, \textbf{segmentar ciertos tejidos sigue siendo un desafío notable}. En particular, las áreas de \textbf{tejido necrótico} suelen ser pequeñas comparadas con el tamaño total de la herida y presentan alta variabilidad (desde costras duras negras hasta material viscoso amarillo), dificultando su reconocimiento consistente por el algoritmo. Incluso con modelos avanzados, los resultados de segmentación para necrosis son significativamente inferiores a los de tejidos como granulación. Chairat  reportaron IoU por debajo de 0,2 para la categoría "necrosis", alineado con otros estudios previos que igualmente encontraron bajos desempeños al segmentar tejido necrótico o esfacelos. Esto se atribuye a varios factores: desequilibrio de clases (pocas muestras de necrosis en los datos de entrenamiento en comparación con abundantes zonas de granulación), la dificultad intrínseca de distinguir visualmente necrosis cubierta de fibrina vs. fondo, y en parte la subjetividad en delinear qué se considera exactamente necrosis (los criterios pueden variar entre médicos, complicando el entrenamiento supervisado). De forma similar, la segmentación del área de epitelización suele presentar desempeño moderado, ya que este tejido nuevo aparece como parches muy delgados en los bordes, a veces difíciles de discernir incluso para el ojo humano y frecuentemente ausentes en estadios tempranos (por lo tanto poco representados en los datos). 

Otra arista crítica es la obtención de las etiquetas de entrenamiento para estos modelos. Requiere que especialistas tracen a mano, sobre cientos de imágenes, contornos precisos de cada tipo de tejido – una tarea ardua y propensa a discrepancias. Estudios han demostrado que la concordancia entre diferentes expertos al segmentar manualmente los tejidos de una herida puede ser limitada. Howell , por ejemplo, encontraron variaciones considerables entre profesionales al identificar las características de una herida en fotografías, reflejando la naturaleza subjetiva del proceso. En el trabajo de Chairat , fue necesario realizar múltiples rondas de discusión y hasta votaciones ciegas entre dermatólogos para acordar un \textit{ground truth} coherente antes de entrenar la red. Por tanto, la validación de las técnicas de segmentación de tejidos debe considerar esta variabilidad: un rendimiento “perfecto” del algoritmo comparado contra la etiqueta de un único experto podría aun así discrepar de la opinión de otro experto. Idealmente, las anotaciones de entrenamiento y prueba se obtienen con consenso de varios especialistas para mitigar el sesgo individual. En resumen, las técnicas modernas permiten abordar la segmentación y clasificación de tejidos en imágenes de heridas de forma automática. La combinación de calibración de imagen, algoritmos de segmentación profunda y un cuidadoso etiquetado experto ha dado lugar a sistemas capaces de identificar las porciones de granulación, necrosis y epitelio con una precisión útil clínicamente. No obstante, persisten áreas de mejora, especialmente en refinar la detección de los tejidos minoritarios y en disponer de bases de datos más amplias y estandarizadas para entrenar/validar estos algoritmos.


\subsection{Validación y desafíos en el uso de IA para la evaluación de heridas}

Si bien los resultados iniciales de la IA en evaluación de heridas son prometedores, su \textbf{traslado a la práctica clínica} enfrenta diversos desafíos que deben ser considerados en el marco conceptual de este proyecto. Uno de los desafíos principales es la \textbf{validación clínica y la confiabilidad} de estas herramientas. A diferencia de aplicaciones puramente técnicas, aquí se trata de sistemas que apoyarían decisiones médicas, por lo que requieren un alto grado de confianza. Esto implica evaluar rigurosamente su desempeño no solo en términos de métricas de imagen (IoU, exactitud de clasificación), sino también en resultados clínicamente relevantes (ej. capacidad para predecir qué heridas no cicatrizarán sin intervención, o concordancia con los juicios de expertos humanos). En este sentido, un obstáculo es la mencionada \textbf{subjetividad del “gold standard”}: la IA típicamente aprende a imitar las evaluaciones humanas, pero cuando estas evaluaciones humanas varían entre profesionales, ¿contra cuál se valida el modelo? Estudios han reportado que la concordancia inter-observador en evaluación de heridas es limitada, con diferencias apreciables en mediciones manuales de tamaño y apreciación de tejido. Esto dificulta definir un punto de referencia firme. Una estrategia para paliar ello, utilizada por Chairat y otros, es involucrar a múltiples expertos en la anotación y utilizar métodos de consenso (votación, promedios) para las etiquetas finales \cite{healthcare11020273}. Aun así, al validar el modelo, es recomendable comparar sus salidas con la opinión de varios especialistas y no de uno solo, para asegurarse de que la herramienta automatizada produzca resultados dentro del rango aceptado por la comunidad clínica. 

Otro desafío es la \textbf{disponibilidad de datos}. Los algoritmos de \textit{deep learning} suelen necesitar un volumen considerable de ejemplos para generalizar bien. En el dominio de heridas, sin embargo, los conjuntos de datos públicos son escasos y relativamente pequeños (a menudo decenas de imágenes, pocas superan el centenar). Además, muchos de ellos solo incluyen la segmentación de área total de la herida y no la anotación de tejidos internos. La recolección de un gran dataset de imágenes de heridas con anotaciones detalladas es complicada por varias razones: protección de datos de pacientes (imágenes clínicas sensibles), variabilidad de presentaciones (se necesitarían casos de pie diabético neuropático, isquémico, mixto; úlceras en distintos estadios; diversas etnias y tonos de piel, etc. para evitar sesgos), y el \textbf{coste en tiempo de los expertos} para etiquetar cada imagen. Este cuello de botella de datos hace que muchos modelos se entrenen con data limitada, corriendo riesgo de sobreajustarse a casos particulares y no rendir bien con imágenes de otros entornos. Incluso el modelo de Curti \cite{Curti2024}, que tuvo buen desempeño, se entrenó y validó con imágenes de un único centro (dataset \textit{DeepSkin} de Italia), lo que deja interrogante sobre su generalización a poblaciones distintas. Por tanto, un reto continuo es \textbf{ampliar y diversificar los datos} de entrenamiento, e idealmente compartir conjuntos de datos de heridas a nivel internacional para impulsar desarrollos más robustos. 

La \textbf{variabilidad en las condiciones de adquisición de las imágenes} es otra fuente de desafíos. A diferencia de otras imágenes médicas (radiografías, TAC) que son bastante estandarizadas, las fotografías de heridas pueden tomarse con diferentes dispositivos (distintos modelos de smartphones o cámaras), bajo variadas condiciones de iluminación ambiente, a distintas distancias o ángulos, etc. Esto impacta directamente en el rendimiento de los algoritmos de visión por computador. Por ejemplo, un modelo entrenado mayoritariamente con imágenes bien iluminadas podría fallar al evaluar una foto tomada en una habitación con poca luz o con luz artificial amarillenta. Chairat reportan que \textbf{diferencias en la iluminación y calibración de color} entre imágenes pueden reducir la precisión de segmentación, ya que los colores del tejido se perciben alterados \cite{healthcare11020273}. Para mitigar este problema, ellos incorporaron una carta de color y tamaño en cada foto, y demostraron que aplicar una \textbf{calibración automática} (ajustando los valores de color de la imagen para que coincidan con los del patrón conocido de la carta) aumentó significativamente las métricas de desempeño del modelo. Este enfoque de normalizar las imágenes es clave para que la IA no dependa de las condiciones específicas de cada foto. Sin embargo, en entornos reales no siempre será posible usar una carta calibradora; por ello, se investiga también que los modelos aprendan a ser invariables a ciertos cambios (por ejemplo, usando técnicas de \textit{data augmentation} que simulen distintas iluminaciones durante el entrenamiento). Otro aspecto es la \textbf{estimación de la profundidad de la herida} a partir de imágenes 2D: esto es prácticamente imposible de forma precisa sin asistencia, y es un parámetro crítico (pues define si una úlcera es superficial o compromete planos profundos). Algunas propuestas para abordar esto incluyen utilizar cámaras con sensor de profundidad o técnicas de fotogrametría para reconstruir un modelo 3D de la herida. Chairat mencionan planes de emplear cámaras 3D de tipo \textit{depth sensor} para captar la estructura volumétrica, o desarrollar dispositivos portátiles que incorporen la IA y sensores en la cabecera del paciente. Si bien esto añade complejidad tecnológica, podría solventar la limitación de las fotos comunes que “aplanan” la información. 

La \textbf{robustez y longevidad} de los algoritmos en el tiempo es otra consideración de validación. Un sistema automático ideal no solo debe funcionar en un solo análisis aislado, sino que debería servir para \textbf{seguimiento longitudinal} de la herida. Es decir, si se toman fotos semanales de una úlcera para ver su progreso, el algoritmo debe ser consistente en sus mediciones a lo largo del tiempo (por ejemplo, si calcula 20\% de necrosis un día y 10\% una semana después, uno espera que ese cambio refleje una realidad y no variaciones aleatorias de la IA). Evaluar la estabilidad longitudinal requiere pruebas específicas, aplicando el modelo en secuencias de tiempo conocidas y verificando que las tendencias que arroja concuerdan con la expectativa clínica. Asimismo, la robustez implica que pequeños cambios irrelevantes (ej., la presencia de una regla desechable al lado de la herida en una foto, o una gasa de fondo) no deberían alterar drásticamente el output del modelo. Muchos de estos aspectos se están abordando mediante técnicas de augmentación, entrenamiento con variedad de escenarios, y módulos de atención en las redes que enfoquen la herida e ignoren contexto no importante. 

Finalmente, existe el desafío de la aceptación e \textbf{integración clínica}. Desde el punto de vista de validación, no basta con demostrar en artículos científicos que un algoritmo funciona; es necesario que las herramientas se integren en el flujo de trabajo real de los profesionales de salud. Esto implica consideraciones de \textbf{interfaz de usuario}, interoperabilidad con historiales clínicos electrónicos, formación del personal, y evidencia de costo-efectividad. Por ejemplo, un algoritmo que provea un puntaje PWAT automatizado debe presentarlo de forma comprensible y útil para el clínico (idealmente junto con la imagen marcada o porcentajes claros), y debe integrarse en la rutina sin añadir mucha carga. También es importante gestionar las expectativas: la IA debe ser vista como soporte a la decisión y no como reemplazo del juicio clínico. Para ganar confianza, algunos sistemas explican sus resultados resaltando las áreas de la imagen que más contribuyeron a la predicción (lo que se conoce como \textbf{IA explicable} o \textit{explainable AI}). En el campo de las heridas, esto podría significar señalar las regiones que la red considera necróticas vs. granulación, para que el médico corrobore visualmente si está de acuerdo. Todos estos factores forman parte de los desafíos para la adopción plena de la IA en la evaluación de heridas.




\section{Estado del Arte}
\label{sc:EA}


Las heridas crónicas, como las úlceras diabéticas del pie , representan un serio problema de salud mundial por su alta prevalencia y complicaciones. Se estima que entre un 19\% y 34\% de los pacientes diabéticos desarrollarán una úlcera de pie en su vida, con elevado riesgo de mala cicatrización, amputación de miembros inferiores e incluso reducción de la supervivencia \cite{Zhang2022}. Las úlceras por presión (escaras) y las úlceras venosas son otros tipos comunes de heridas crónicas, todas contribuyendo a costos sanitarios multimillonarios \cite{Wang2020,Sendilraj2024}. La evaluación clínica de estas lesiones suele basarse en la inspección visual y mediciones manuales, procesos sujetos a la experiencia del profesional y, por tanto, con variabilidad interobservador \cite{Curti2024}. Para estandarizar la valoración, se han propuesto escalas clínicas como \textbf{BWAT}, \textbf{PUSH} o la clasificación de \textbf{Wagner}, entre otras. Sin embargo, aplicarlas requiere tiempo y puede ser subjetivo. En la última década, la \textbf{inteligencia artificial} (IA) – especialmente técnicas de \textbf{machine learning} (ML) y \textbf{deep learning} (DL) – ha emergido como herramienta prometedora para automatizar la evaluación de heridas ulcerosas, buscando mayor objetividad, eficiencia y seguimiento continuo \cite{Sendilraj2024,Curti2024}.

A continuación, se presenta una revisión crítica de trabajos relevantes (2015–2025) sobre el uso de IA en la evaluación automática de heridas ulcerosas, con énfasis en pie diabético pero sin limitarse a este. Se abordan desarrollos en segmentación de imágenes (delimitación de la herida y análisis de tamaño), clasificación y gradación de severidad, análisis de tejidos en la herida, predicción de evolución de la cicatrización, así como enfoques multimodales. Se comparan metodologías, resultados, conjuntos de datos empleados, fortalezas, debilidades y posibles mejoras, organizando la información en secciones temáticas para facilitar su lectura.

\subsection{Escalas clínicas de evaluación de heridas}

En la práctica clínica, existen escalas estandarizadas para evaluar el estado y la evolución de heridas. Por ejemplo, \textbf{BWAT (Bates-Jensen Wound Assessment Tool)} consta de 13 ítems que evalúan características como:
\begin{enumerate}
    \item tamaño
    \item profundidad de la herida
    \item bordes
    \item presencia de socavamiento
    \item tipo de tejido necrotico
    \item cantidad de tejido necrotico
    \item cantidad de tejido de ganulacion
    \item cantidad de tejido de epilizacion
    \item tipo de exudado
    \item cantidad de exudad
    \item condicion de la piel peilesional - color
    \item condicion de la piel peilesional - edema
    \item condicion de la piel peilesional - induracion
    
\end{enumerate}
Cada ítem se puntúa de 1 (mejor situación) a 5 (peor), ofreciendo un puntaje total que cuantifica la gravedad de la herida. Debido a que varios ítems requieren examen directo (ej., medir profundidad o socavamientos manualmente), el \textbf{BWAT} tradicionalmente se aplica \textbf{en persona} durante la visita clínica . Esto dificulta su automatización posterior a partir de fotografías, pues algunas características no son discernibles solo con imágenes. Para solventar esa limitación, se introdujo la escala \textbf{PWAT} \cite{Curti2024}. 

El PWAT es una adaptación del BWAT que incluye solo un subconjunto de los ítems –aquellos inferibles directamente de una fotografía–, excluyendo por ejemplo la profundidad o socavamientos. Pese a ser más limitado, el PWAT ha demostrado \textbf{validez y robustez} en aplicaciones clínicas para seguimiento mediante imágenes . Otra escala común es la \textbf{PUSH (Pressure Ulcer Scale for Healing)}, diseñada por NPIAP para monitorizar la cicatrización de úlceras por presión, combinando medidas de área superficial de la herida, cantidad de exudado y tipo de tejido dominante en el lecho ulceroso (granulación, esfacelo/fibrina o epitelial). La PUSH genera un puntaje total que disminuye a medida que la herida cura, facilitando evaluar si una úlcera por presión está mejorando o no

En cuanto a clasificación de severidad, para las úlceras de pie diabético se emplea la \textbf{clasificación de Wagner}, de 6 niveles (0 a 5) que describen la profundidad y complicaciones de la lesión . Un Wagner 0 indica pie de riesgo sin ulceración; Wagner 1 es una úlcera superficial; Wagner 2 implica extensión a tendón, hueso o cápsula articular; Wagner 3 añade infección grave (ej. osteomielitis o absceso); Wagner 4 denota gangrena localizada (por isquemia/infección) en antepié o talón; y Wagner 5 es gangrena extensa de todo el pie, generalmente requiriendo amputación \cite{Girmaw2025}. Esta clasificación guía el manejo clínico del pie diabético y pronostica la necesidad de intervenciones mayores.

Aunque las escalas estandarizadas han reducido parcialmente la subjetividad, sigue habiendo variabilidad entre observadores al aplicarlas. Por ejemplo, dos médicos podrían discrepar en la estimación del porcentaje de tejido de granulación vs. esfacelo en un lecho ulceroso, afectando la puntuación BWAT. En efecto, la evaluación en forma de escalas Likert impone categorías que \textbf{fuerzan la cuantificación} de rasgos de la herida, pero la interpretación visual sigue siendo del clínico . Estudios reportan que persiste una variabilidad inter e intraobservador no desdeñable incluso utilizando herramientas estandarizadas . Esto ha motivado la investigación en sistemas automatizados que extraigan de imágenes métricas objetivas de la herida (dimensiones, tipos de tejido, etc.) para calcular o apoyar estas escalas. A continuación se revisan los avances en dichos sistemas basados en IA.

Segmentar una herida en una imagen consiste en delinear automáticamente el contorno de la lesión y separarla de la piel sana circundante. Es un paso fundamental, pues permite calcular la superficie de la úlcera, un parámetro clave para monitorear la cicatrización (una reducción del área a lo largo del tiempo suele indicar progreso) \cite{Wang2020}. La medición manual del área (ej., con reglas o calcos sobre la herida) es engorrosa, imprecisa y consume tiempo \cite{Filko2023}. Automatizar la segmentación permite obtener \textbf{medidas objetivas de forma rápida}, facilitando la documentación y seguimiento en historias clínicas electrónicas . 

\begin{itemize}
    \item \textbf{Enfoques tradicionales}
    \begin{itemize}
        \item Antes del auge del deep learning, se exploraron métodos de visión por computador clásica para segmentar heridas crónicas. Estos incluían técnicas como umbralización de color, detección de bordes, crecimiento de regiones y uso de clasificadores basados en características diseñadas manualmente. Por ejemplo, Wang \cite{Chemello2022} propusieron un método de dos etapas: primero se sobresegmenta la imagen en superpixels y se extraen múltiples descriptores de color y textura; luego, un conjunto de clasificadores SVM distingue superpixels de herida vs. piel . Este clasificador en cascada logró sensibilidad $\sim73.3\%$ y especificidad $\sim94.6\%$ en segmentar úlceras diabéticas (100 imágenes de 15 pacientes) , superando a enfoques de una sola etapa o redes neuronales poco profundas probadas en esa cohorte. Posteriormente, Wang \cite{Chemello2022}  abordaron el reto de la variabilidad en condiciones de captura (iluminación, ángulo, fondo) entrenando un modelo de Random Field jerárquico (AHRF) que extendía los Conditional Random Fields. Probado en un conjunto pequeño de imágenes sintéticas y reales, este modelo logró especificidad >95\% y sensibilidad >77\% , mostrando mejor desempeño que CRFs tradicionales. Los autores sugirieron que en escenarios con muy pocos datos, modelos probabilísticos como AHRF podrían superar a CNNs profundas, aunque anticiparon que con más datos las redes neuronales lograrían resultados superiores .
    \end{itemize}
    \item \textbf{Deep learning en segmentación}
    \begin{itemize}
        \item A medida que aumentaron los conjuntos de datos de heridas, los métodos de deep learning demostraron ventajas claras en segmentación. Diversos arquitecturas de segmentación semántica han sido aplicadas: Fully Convolutional Networks, U-Net, SegNet, etc. Ohura evaluaron redes para segmentar úlceras combinando datos de distintos tipos de heridas: entrenaron CNNs (SegNet, LinkNet, U-Net) principalmente con 400 imágenes de úlceras por presión y apenas 20 imágenes de UPD \cite{Chemello2022}. Sorprendentemente, la U-Net logró altísima precisión al segmentar incluso las úlceras diabéticas (especificidad 0.943 y sensibilidad 0.993 en promedio) , evidenciando que patrones aprendidos en úlceras por presión podían transferirse a otras etiologías crónicas con morfologías parecidas. Este resultado sugirió que, dadas suficientes imágenes de entrenamiento, las redes profundas segmentan heridas con fiabilidad casi humana
        
        \item Varios grupos han construido sus propios datasets para entrenar modelos. Wang \cite{Wang2020} compilaron 1109 imágenes de úlceras de pie diabético de 889 pacientes y entrenaron un modelo basado en MobileNetV2 para segmentación automática . Eligiendo MobileNetV2 por su ligereza, demostraron que un modelo de baja complejidad podía lograr un rendimiento equiparable a redes más profundas en esa tarea . La arquitectura propuesta combinaba la segmentación por CNN con un post-procesamiento de componentes conexos para afinar la máscara final . Gracias a su menor costo computacional, este enfoque apuntaba a implementaciones en dispositivos móviles sin sacrificar precisión. De hecho, otras investigaciones también resaltan la factibilidad de ejecutar modelos de segmentación en smartphones: Ramachandram \cite{Ramachandram2022} entrenaron modelos U-Net con el mayor conjunto publicado hasta la fecha ($\approx465$ mil pares de imagen-máscara para segmentar heridas, más 17 mil para segmentar tejidos dentro de la herida) obtenidos de la base de datos de la compañía Swift Medical . El modelo resultante segmenta en tiempo casi real en un teléfono, dada la optimización lograda con ese enorme dataset . En pruebas, su red alcanzó un IoU promedio de 0,8644 delimitando el área de la herida (muy alto acuerdo con el contorno verdadero). Incluso en condiciones de iluminación y piel variadas, la segmentación automática mostró ser robusta y sin sesgos por tono de piel – un aspecto importante para equidad clínica

        \item La precisión de la segmentación profunda se refleja también en desafíos internacionales. En la competencia Diabetic Foot Ulcer Challenge (DFUC2020), múltiples equipos aplicaron detectores y segmentadores basados en YOLO, Faster R-CNN y U-Net. Un resumen de Yap et al. reportó que las mejores refinaciones de YOLOv3 lograron $\sim 91.95\%$ de exactitud en detección de úlcera en imagen completa, y variantes de Faster R-CNN alcanzaron hasta 91.4\% mAP . Para segmentación semántica, arquitecturas tipo U-Net destacaron; en un estudio se informa que un modelo U-Net superó a otras arquitecturas con 94.96\% de precisión en segmentación de la herida . Asimismo, aplicando Mask RCNN (que combina detección y segmentación a nivel de instancia) se han logrado valores de precision $\sim 0.86$ y mAP $\sim 0.51$ segmentando úlceras . En suma, la comunidad ha validado que las CNN bien entrenadas pueden delimitar las heridas con alta fiabilidad, permitiendo calcular el área de forma automática y consistente
    \end{itemize}
    \item \textbf{Comparativa de enfoques de segmentación}
    \begin{itemize}
        \item Los métodos basados en DL superan a los clásicos en exactitud, especialmente cuando hay suficiente volumen de datos para entrenar. Los enfoques tradicionales (p. ej. SVM con atributos diseñados) alcanzaron especificidades altas en conjuntos pequeños , pero su sensibilidad quedaba limitada posiblemente por la variabilidad visual que no capturaban totalmente. Las CNN, al aprender características de bajo a alto nivel directamente de los píxeles, manejan mejor dicha variabilidad, logrando sensibilidades y especificidades cercanas al 90–99\% \cite{Chemello2022}. No obstante, un desafío común es la \textbf{generalización}: muchos estudios emplearon datos de un único centro o condiciones controladas, y un modelo puede perder precisión si se aplica a imágenes con distinta iluminación, dispositivos o poblaciones. Para mitigar esto, algunos autores integran pasos de pre-procesamiento de color (ej. conversión a espacios de color uniformes Lab, YCbCr) antes de la segmentación, como en la plataforma DFUCare \cite{Sendilraj2024}, o aplican extensas técnicas de \textit{data augmentation } \cite{Aldughayfiq2023} para expandir la diversidad de ejemplos de entrenamiento. Otra tendencia es combinar modalidades: Filko \cite{Filko2023} desarrollaron un sistema robótico que captura simultáneamente la herida en 2D (fotografía RGB) y en 3D mediante escaneo láser, usando una CNN 2D para segmentar inicialmente la herida en la imagen y luego refinando el contorno sobre la malla 3D con un modelo activo de contornos \cite{Filko2023}. Este enfoque híbrido produce un modelo 3D de la superficie herida, permitiendo medir \textbf{perímetro, área y volumen} de la úlcera de forma totalmente automática . La incorporación de volumen es valiosa, ya que una reducción volumétrica puede indicar curación antes que la reducción de área en ciertas lesiones profundas.
    \end{itemize}
\end{itemize}

\subsection{Clasificación de heridas: detección, severidad e infección}

Otra línea de aplicación de la IA en heridas es la clasificación automática, que puede tomar varias formas según el objetivo clínico:
\begin{itemize}
    \item \textbf{Detección vs. piel sana}
    \begin{itemize}
        \item Identificar si en una foto de un pie o de la piel hay una úlcera presente (clasificación binaria herida/no herida)
    \end{itemize}
    \item \textbf{Clasificación del tipo de herida}
    \begin{itemize}
        \item  Por etiología (diabética, venosa, por presión, quirúrgica, etc.) o por estadio (p. ej. estadios I–IV de úlcera por presión, grados Wagner 0–5 en pie diabético).
    \end{itemize}
    \item \textbf{Detección de signos clínicos en la herida}
    \begin{itemize}
        \item clasificar si una úlcera de pie diabético muestra signos de infección, de isquemia, ambas o ninguna (clasificación multinomial)
    \end{itemize}
    \item \textbf{Gradación de severidad}
    \begin{itemize}
        \item Asociada a escalas como Wagner, Texas, NPUAP, etc., a partir de la imagen.
    \end{itemize}
\end{itemize}

Varios trabajos han explorado redes neuronales para estas tareas de clasificación a nivel de imagen o de herida. En general, las arquitecturas de deep learning utilizadas son CNNs de clasificación (VGG, ResNet, EfficientNet, MobileNet, etc.) o detectores de objetos cuando se necesita localizar la herida en la foto además de clasificarla

\begin{itemize}
    \item \textbf{Detección de úlceras (binario sí/no)}
    \begin{itemize}
        \item Goyal \cite{Goyal2020} desarrollaron \textbf{DFUNet}, una arquitectura de CNN especializada en identificar regiones con úlcera vs piel normal en el pie diabético . DFUNet combinaba convoluciones profundas con capas paralelas y de distinta profundidad para captar características a múltiples escalas, logrando mejorar la discriminación entre piel sana y lesionada . Este trabajo pionero ya demostraba la viabilidad de deep learning para detectar automáticamente una úlcera en imágenes donde pudiera haber artefactos o variabilidad en los pies de pacientes diabéticos. Estudios posteriores han reportado precisiones muy altas en esta tarea: Cassidy \cite{Cassidy2023} hicieron un \textbf{estudio clínico multicéntrico} con 203 fotografías de pies tomadas con un smartphone de bajo costo, pasando cada imagen por un sistema de IA para detectar úlceras y comparando con la evaluación de especialistas. El modelo alcanzó \textbf{91.6\% de sensibilidad y 92.4\% de especificidad} en la detección de úlceras , prácticamente equiparable a la precisión humana. Es notable que incluso con imágenes capturadas en entornos reales (no de estudio), la IA mantuvo alto desempeño, lo cual avala su potencial uso como herramienta de tamizaje remoto. Los clínicos participantes mostraron además alto acuerdo (Kappa >0.8) en que el sistema identifica correctamente las úlceras . Este es uno de los primeros ensayos que prueban un algoritmo de visión computacional en campo con pacientes, marcando un hito hacia la adopción como dispositivo médico

        \item En escenarios más controlados, se han conseguido incluso accuracies cercanas a la perfección en detección binaria, aunque en conjuntos de datos limitados. Por ejemplo, Girmaw y Taye \cite{Girmaw2025} entrenaron un modelo MobileNetV2 para detectar la presencia de UPD y reportan \textbf{100\% de exactitud} en sus pruebas . Si bien este resultado es llamativo, es probable que su dataset no haya sido muy amplio ni diverso, lo que podría indicar cierto \textit{overfitting}. Aun así, refleja la capacidad de las CNN modernas: un modelo ligero fue suficiente para separar completamente imágenes con vs sin úlcera en el conjunto evaluado . Los autores destacan la importancia de la cuidadosa preparación de datos (anotaciones, aumentos, etc.) y ajuste de hiperparámetros para lograr tal rendimiento , subrayando que la clave está tanto en la arquitectura como en el \textit{fine-tuning} adecuado.
    \end{itemize}
        
\end{itemize}



\chapter{Definición del Problema }
\label{ch:Problema}
\usetikzlibrary{arrows,positioning}
\section{Definición del Problema}
\subsection{Problema}
\label{ssec:P}

La evaluación clínica actual del pie diabético está predominantemente basada en criterios visuales subjetivos, lo que genera importantes inconsistencias diagnósticas y retrasa la implementación de intervenciones clínicas oportunas y efectivas. Diversas investigaciones han señalado que esta subjetividad genera una significativa variabilidad inter e intraoperador, afectando negativamente el pronóstico y evolución clínica de los pacientes \cite{mishra2017diabetic, bandyk2018diabetic, thompson2013reliability}. Aunque existen herramientas como el Photographic Wound Assessment Tool (PWAT), diseñadas para reducir dicha subjetividad y mejorar la precisión diagnóstica mediante la evaluación de imágenes, aún persisten limitaciones en términos de implementación práctica, objetividad plena y automatización del proceso \cite{thompson2013reliability, organizacion2016informe}.

Recientemente, se han desarrollado diversos enfoques automatizados basados en inteligencia artificial y técnicas avanzadas de procesamiento de imágenes para mejorar la precisión y objetividad de estas evaluaciones. Sin embargo, muchas de estas propuestas todavía enfrentan dificultades relacionadas con la estandarización del proceso, la integración efectiva de información clínica adicional y la necesidad de dispositivos específicos o condiciones altamente controladas para la captura de imágenes \cite{van2017computational}. Un ejemplo relevante es el trabajo reciente de Curti \cite{Curti2024}, quienes presentaron una propuesta automatizada para la predicción del puntaje PWAT utilizando imágenes capturadas con smartphones y técnicas avanzadas de análisis radiómico y aprendizaje automático, logrando una alta correlación con las evaluaciones clínicas manuales \cite{Curti2024}. A pesar de este avance, se evidencian limitaciones como la dependencia de un conjunto limitado y monocéntrico de imágenes, la ausencia de condiciones estandarizadas de captura y la dificultad para una integración clínica directa y sencilla.

\subsection{Solución Propuesta}
\label{ssec:SP}

Considerando los hallazgos de la literatura y las limitaciones existentes, la solución propuesta en este trabajo se enfoca en desarrollar una herramienta automatizada robusta que supere dichas restricciones. Para esto, se integrará un modelo avanzado de segmentación automática de heridas ya desarrollado, capaz de identificar y delimitar con precisión el área de las lesiones y las zonas circundantes (peri‑lesión). Estas segmentaciones serán la base para extraer características radiómicas altamente informativas mediante técnicas especializadas como pyradiomics. Dichas características incluyen parámetros relacionados con la textura, morfología, color y variabilidad de la herida, elementos que tradicionalmente los clínicos evalúan de forma visual y subjetiva.

Posteriormente, se aplicarán técnicas avanzadas de aprendizaje automático (machine learning), con el fin de obtener estimaciones precisas y objetivas del puntaje PWAT. Se contemplarán diferentes algoritmos predictivos, con el propósito de seleccionar la aproximación más efectiva. Además, se integrará información clínica relevante proveniente de registros digitales del paciente, tales como datos demográficos, historial clínico, parámetros de laboratorio y otras variables asociadas a la condición clínica del individuo. La integración de estos datos adicionales permitirá mejorar considerablemente la precisión diagnóstica y asegurar una evaluación integral del paciente, facilitando así la toma de decisiones clínicas oportunas y fundamentadas en evidencia.

Finalmente, esta herramienta se implementará en una interfaz de usuario amigable, diseñada para ser fácilmente adoptada en la práctica clínica cotidiana, minimizando las barreras tecnológicas y asegurando una aplicabilidad clínica directa y efectiva.

\subsection{Importancia del trabajo}
\label{ssec:IT}

Desde un punto de vista científico y social, el presente trabajo es altamente relevante. Científicamente, este proyecto ofrece una significativa contribución en la integración y aplicación práctica de métodos avanzados como machine learning, análisis radiómico y procesamiento automático de imágenes en el ámbito clínico, especialmente en el manejo de heridas crónicas asociadas a diabetes. La aplicación efectiva de estas técnicas permitirá establecer una metodología de evaluación de heridas mucho más precisa y reproducible, lo cual representa un avance significativo frente a las técnicas tradicionales.

Desde la perspectiva social, la importancia radica en su potencial para mejorar considerablemente la calidad de vida de los pacientes con pie diabético, reduciendo el riesgo de complicaciones graves como infecciones recurrentes y amputaciones, que impactan profundamente en la autonomía y bienestar de los afectados. Además, al optimizar la precisión diagnóstica y promover intervenciones clínicas oportunas, esta herramienta podría reducir los costos sanitarios derivados de tratamientos prolongados y hospitalizaciones recurrentes, facilitando una gestión más eficiente de los recursos del sistema de salud. En suma, el desarrollo e implementación de esta propuesta contribuirá significativamente a un manejo clínico más eficaz, centrado en el paciente y basado en evidencia científica robusta, con beneficios directos para individuos, comunidades y sistemas sanitarios en general.







\chapter{Objetivos}
\label{ch:OG}
\input{objetivos}

\chapter{Metodologia}
\label{ch:Met}
\input{metodologia}

\section{Planificación}
\label{ch:Plan}
\input{planificacion}

\chapter{Especificación de Requerimientos}
\label{ch:req}
\section{Especificación de Requerimientos}
\label{sc:ER}

Este sistema tiene como propósito apoyar al personal clínico en la estimación automática del estadio de heridas ulcerosas en pacientes con pie diabético, utilizando imágenes médicas procesadas mediante técnicas de segmentación, extracción de características radiómicas y modelos de predicción basados en aprendizaje automático.


\subsection{Requerimientos Funcionales}
\label{ssc:RF}

\begin{itemize}
    \item \textbf{RF1: Registro de usuario.} El sistema debe permitir registrar nuevos usuarios en la tabla \texttt{user}, incluyendo su RUT, nombre, correo, contraseña, rol y fecha de nacimiento.

    \item \textbf{RF2: Gestión de pacientes.} El sistema debe permitir al personal clínico registrar y actualizar información de pacientes en la tabla \texttt{paciente}, asociando cada paciente a un usuario y a un profesional responsable.

    \item \textbf{RF3: Registro de profesionales.} El sistema debe permitir registrar y administrar profesionales clínicos en la tabla \texttt{profesional}, vinculando cada uno a un usuario registrado.

    \item \textbf{RF4: Carga de imágenes clínicas.} El sistema debe permitir subir imágenes de heridas ulcerosas, guardando su nombre, fecha de captura, ruta de almacenamiento y asociación con el paciente correspondiente en la tabla \texttt{imagen}.

    \item \textbf{RF5: Segmentación de imágenes.} El sistema debe permitir registrar los resultados de segmentación de una imagen, especificando el método usado (manual o automática), la ruta de la máscara y la fecha de creación, en la tabla \texttt{segmentacion}.

    \item \textbf{RF6: Evaluación del estadio de la herida.} El sistema debe permitir registrar el puntaje PWAT calculado por modelo o experto humano en la tabla \texttt{pwatscore}, vinculando los resultados con una imagen y su segmentación correspondiente.

    \item \textbf{RF7: Visualización de evaluaciones.} El sistema debe permitir consultar el historial de evaluaciones PWAT asociadas a un paciente, visualizando fecha, evaluador, observaciones, imagen relacionada y resultados por categoría (JSON).

    \item \textbf{RF8: Validación cruzada de evaluaciones.} El sistema debe permitir comparar las evaluaciones realizadas por el modelo con aquellas realizadas por expertos clínicos sobre la misma imagen.

    \item \textbf{RF9: Gestión de acceso por rol.} El sistema debe restringir el acceso a funcionalidades según el campo \texttt{rol} de la tabla \texttt{user}, permitiendo, por ejemplo, que sólo usuarios con rol \texttt{doctor} o \texttt{enfermera} puedan ingresar evaluaciones clínicas.

\end{itemize}



\subsection{Requerimientos No Funcionales}
\label{ssc:RNF}

\begin{itemize}
    \item \textbf{RNF1:} El sistema debe procesar una imagen y mostrar el resultado en un tiempo máximo de 5 segundos.
    
    \item \textbf{RNF2:} El sistema debe ser portable y permitir escalabilidad horizontal mediante contenedores Docker, de forma que se puedan ejecutar multiples instancias concurrentes
    
    \item \textbf{RNF3:} El sistema debe tener una disponibilidad mínima del 99.5\% mensual.
    
    \item \textbf{RNF4:} El sistema debe utilizar cifrado TLS 1.2 o superior para la transmisión de imágenes y datos clínicos.
    
    \item \textbf{RNF5:} La interfaz de usuario debe ser intuitiva y responsiva, compatible con dispositivos móviles y navegadores modernos.
    
    \item \textbf{RNF6:} El código del sistema debe estar modularizado y documentado, facilitando su mantenimiento evolutivo y correctivo.
    
    \item \textbf{RNF7:} El sistema debe ser compatible con las tres últimas versiones de los navegadores Chrome, Firefox, Safari y Edge.
    
    \item \textbf{RNF8:} El backend debe permitir la interoperabilidad entre servicios desarrollados en Node.js y scripts Python, mediante APIs REST o invocaciones asincrónicas.
    
    \item \textbf{RNF9:} Cada predicción debe ser registrada con un identificador único y metadatos asociados (timestamp, parámetros utilizados, resultado) para permitir trazabilidad clínica y técnica.
\end{itemize}


\section{Funcionalidades del Sistema}
\label{sc:FS}


\subsection{Diagramas de Casos de Uso}
\label{ssc:DCU}

\begin{figure}[H] %si cambia la "H" por "t", se fuerza a que la figura quede arriba de la página
    \centering
    \includegraphics[width=1.1\textwidth]{imagenes/casosdeuso.drawio.png}
    \caption{Diagrama de casos de uso}
    \label{fig:scrum}
\end{figure}


\subsection{Casos de Uso (resumidos)}
\label{ssc:CUresumido}

\begin{table}[H]
  \small
  \centering
  \caption{Resumen condensado de Casos de Uso}
  \label{tab:cu_resumido}
  \begin{tabular}{p{1cm} p{6cm} p{2cm} p{4cm}}
    \toprule
    \textbf{ID} & \textbf{Descripción}                          & \textbf{Actor(es)}           & \textbf{Relación}           \\
    \midrule
    CU01        & Carga imagen de la úlcera                     & Profesional                  & Extiende CU02               \\
    CU02        & Extrae características radiomédicas           & Profesional                  & Depende de CU01             \\
    CU03        & Segmenta la imagen de la úlcera               & Profesional                  & Depende de CU01             \\
    CU04        & Edita segmentación automática                 & Profesional                  & Extiende CU03               \\
    CU05        & Visualiza resultados (PWAT)                   & Profesional, Paciente        & Incluye CU03, CU06          \\
    CU06        & Calcula y guarda puntaje PWAT                 & Profesional                  & Depende de CU03             \\
    CU07        & Consulta historial de evaluaciones            & Profesional, Paciente        & Depende de CU06             \\
    CU08        & Gestiona registro de paciente                 & Administrador                & –                           \\
    CU09        & Gestiona usuarios                             & Administrador                & –                           \\
    CU10        & Ajusta parámetros del modelo predictivo       & Investigador                 & Depende de CU06             \\
    CU11        & Evalúa desempeño del modelo                   & Investigador                 & Depende de CU06             \\
    \bottomrule
  \end{tabular}
\end{table}



\subsection{Especificación de Casos de Uso}
\label{ssc:CUEspec}

\paragraph{CU01 -- Cargar Imagen Úlcera}
\begin{itemize}
  \item \textbf{Descripción:} Permite al profesional subir o capturar la fotografía clínica de la herida ulcerosa.
  \item \textbf{Actor(es):} Profesional
  \item \textbf{Dependencias:} — (caso de uso inicial)
\end{itemize}

\paragraph{CU02 -- Extraer Características Radiomédicas}
\begin{itemize}
  \item \textbf{Descripción:} Calcula automáticamente descriptores de textura, forma y color sobre la región segmentada de la herida (radiomics).
  \item \textbf{Actor(es):} Profesional
  \item \textbf{Depende de:} CU01
\end{itemize}

\paragraph{CU03 -- Segmentar Imagen Úlcera}
\begin{itemize}
  \item \textbf{Descripción:} Genera automáticamente las máscaras que delimitan la herida y los diferentes tejidos (granulación, necrosis, epitelización).
  \item \textbf{Actor(es):} Profesional
  \item \textbf{Depende de:} CU01
\end{itemize}

\paragraph{CU04 -- Editar Segmentación Automática}
\begin{itemize}
  \item \textbf{Descripción:} Permite al profesional corregir manualmente cualquier imprecisión en las máscaras obtenidas automáticamente.
  \item \textbf{Actor(es):} Profesional
  \item \textbf{Extiende a:} CU03
\end{itemize}

\paragraph{CU05 -- Visualizar Resultados (PWAT)}
\begin{itemize}
  \item \textbf{Descripción:} Muestra al profesional y al paciente la imagen original, las máscaras de segmentación, los valores de radiomics y el puntaje PWAT calculado.
  \item \textbf{Actor(es):} Profesional, Paciente
  \item \textbf{Incluye:} CU03, CU06
\end{itemize}

\paragraph{CU06 -- Predecir Puntaje PWAT}
\begin{itemize}
  \item \textbf{Descripción:} Invoca el modelo predictor para calcular y almacenar el puntaje PWAT sobre la imagen segmentada.
  \item \textbf{Actor(es):} Profesional
  \item \textbf{Depende de:} CU03
\end{itemize}

\paragraph{CU07 -- Consultar Evaluaciones Anteriores}
\begin{itemize}
  \item \textbf{Descripción:} Permite revisar el historial completo de puntajes PWAT asociados a un paciente o a una imagen concreta.
  \item \textbf{Actor(es):} Profesional, Paciente
  \item \textbf{Depende de:} CU06
\end{itemize}

\paragraph{CU08 -- Registrar Paciente}
\begin{itemize}
  \item \textbf{Descripción:} Crea o actualiza el expediente de un paciente en la base de datos (\texttt{paciente}).
  \item \textbf{Actor(es):} Administrador
  \item \textbf{Dependencias:} — (caso de uso independiente)
\end{itemize}

\paragraph{CU09 -- Administrar Usuarios}
\begin{itemize}
  \item \textbf{Descripción:} Permite dar de alta, baja, modificar y consultar usuarios (profesionales o investigadores) en la tabla \texttt{user}.
  \item \textbf{Actor(es):} Administrador
  \item \textbf{Dependencias:} — (caso de uso independiente)
\end{itemize}

\paragraph{CU10 -- Ajustar Parámetros de Modelos}
\begin{itemize}
  \item \textbf{Descripción:} Permite al investigador reentrenar el modelo de predicción o ajustar sus hiperparámetros.
  \item \textbf{Actor(es):} Investigador
  \item \textbf{Depende de:} CU06
\end{itemize}

\paragraph{CU11 -- Evaluar Desempeño de Modelos}
\begin{itemize}
  \item \textbf{Descripción:} Ejecuta métricas de calidad (IoU, correlación de Spearman, etc.) sobre el conjunto de prueba para validar el modelo.
  \item \textbf{Actor(es):} Investigador
  \item \textbf{Depende de:} CU06
\end{itemize}

\subsection{Diagramas de Secuencia}
\label{ssc:DSS}

A continuación se presentan los diagramas de secuencia para cada uno de los once casos de uso definidos en la sección~\ref{ssc:CUresumido}. Cada figura muestra la interacción entre los actores y los componentes del sistema, paso a paso, desde la invocación de la funcionalidad hasta su respuesta final.

\begin{figure}[H]
  \centering
  \includegraphics[width=0.8\textwidth]{imagenes/cu01_seq.png}
  \caption{CU01 – Cargar Imagen Úlcera. El Profesional sube o captura la fotografía clínica y el sistema la persiste en la base de datos de imágenes.}
  \label{fig:cu01_seq}
\end{figure}

\begin{figure}[H]
  \centering
  \includegraphics[width=0.8\textwidth]{imagenes/cu02_seq.png}
  \caption{CU02 – Extraer Características Radiomédicas. Después de la carga, el Profesional solicita el cálculo de descriptores radiómicos y el servicio asociado retorna los resultados.}
  \label{fig:cu02_seq}
\end{figure}

\begin{figure}[H]
  \centering
  \includegraphics[width=0.8\textwidth]{imagenes/cu03_seq.png}
  \caption{CU03 – Segmentar Imagen Úlcera. El Profesional invoca la segmentación automática, que produce la(s) máscara(s) de herida y tejidos.}
  \label{fig:cu03_seq}
\end{figure}

\begin{figure}[H]
  \centering
  \includegraphics[width=0.8\textwidth]{imagenes/cu04_seq.png}
  \caption{CU04 – Editar Segmentación Automática. El Profesional corrige manualmente la máscara generada y almacena la versión ajustada.}
  \label{fig:cu04_seq}
\end{figure}

\begin{figure}[H]
  \centering
  \includegraphics[width=0.8\textwidth]{imagenes/cu05_seq.png}
  \caption{CU05 – Visualizar Resultados. Profesional o Paciente recuperan y muestran la imagen, las máscaras y los puntajes PWAT generados.}
  \label{fig:cu05_seq}
\end{figure}

\begin{figure}[H]
  \centering
  \includegraphics[width=0.8\textwidth]{imagenes/cu06_seq.png}
  \caption{CU06 – Predecir Puntaje PWAT. El Profesional solicita la predicción, se segmenta internamente y el Modelo Predictor devuelve el puntaje, que se guarda en la BD.}
  \label{fig:cu06_seq}
\end{figure}

\begin{figure}[H]
  \centering
  \includegraphics[width=0.8\textwidth]{imagenes/cu07_seq.png}
  \caption{CU07 – Consultar Evaluaciones Anteriores. Profesional o Paciente consultan el histórico de puntajes PWAT filtrados por paciente o imagen.}
  \label{fig:cu07_seq}
\end{figure}

\begin{figure}[H]
  \centering
  \includegraphics[width=0.8\textwidth]{imagenes/cu08_seq.png}
  \caption{CU08 – Registrar Paciente. El Administrador crea o actualiza registros en la tabla \texttt{paciente} mediante la interfaz.}
  \label{fig:cu08_seq}
\end{figure}

\begin{figure}[H]
  \centering
  \includegraphics[width=0.8\textwidth]{imagenes/cu09_seq.png}
  \caption{CU09 – Administrar Usuarios. El Administrador realiza operaciones CRUD sobre la tabla \texttt{user}.}
  \label{fig:cu09_seq}
\end{figure}

\begin{figure}[H]
  \centering
  \includegraphics[width=0.8\textwidth]{imagenes/cu10_seq.png}
  \caption{CU10 – Ajustar Parámetros de Modelos. El Investigador lanza un nuevo entrenamiento o ajuste de hiperparámetros y recibe métricas.}
  \label{fig:cu10_seq}
\end{figure}

\begin{figure}[H]
  \centering
  \includegraphics[width=0.8\textwidth]{imagenes/cu11_seq.png}
  \caption{CU11 – Evaluar Desempeño de Modelos. El Investigador solicita evaluación sobre test‐set y el sistema retorna métricas (IoU, Spearman, etc.).}
  \label{fig:cu11_seq}
\end{figure}

\subsection{Diagramas de Estado}
\label{ssc:DE}

A continuación se presentan los diagramas de estado para los tres elementos clave del sistema: \texttt{Imagen}, \texttt{Segmentación} y \texttt{PwatScore}. Cada figura muestra los estados por los que atraviesa el objeto y las transiciones disparadas por los casos de uso correspondientes.

\begin{figure}[H]
  \centering
  \includegraphics[width=0.95\textwidth,height=0.5\textheight,keepaspectratio]{imagenes/estado_imagen.png}
  \caption{Diagrama de estado de la entidad \texttt{Imagen}.}
  \label{fig:estado_imagen}
\end{figure}

\begin{figure}[H]
  \centering
  \includegraphics[width=0.95\textwidth,height=0.5\textheight,keepaspectratio]{imagenes/estado_segmentacion.png}
  \caption{Diagrama de estado de la entidad \texttt{Segmentación}.}
  \label{fig:estado_segmentacion}
\end{figure}

\begin{figure}[H]
  \centering
  \includegraphics[width=0.95\textwidth,height=0.5\textheight,keepaspectratio]{imagenes/estado_pwatscore.png}
  \caption{Diagrama de estado de la entidad \texttt{PwatScore}.}
  \label{fig:estado_pwatscore}
\end{figure}

\subsection{Modelo Conceptual}
\label{ssc:MC}

A continuación se describe el modelo conceptual (diagrama de entidades y relaciones) que organiza las principales entidades del sistema y sus interacciones:

\begin{itemize}
  \item \textbf{User}: representa a todos los usuarios de la plataforma (clínicos, administradores e investigadores). Contiene atributos de identificación, datos de contacto y rol.
  \item \textbf{Profesional}: extiende a \texttt{User} con la especialidad y vincula al médico o enfermero responsable de pacientes.
  \item \textbf{Paciente}: almacena la información demográfica y clínica básica, referenciando al \texttt{User} que lo registró y al \texttt{Profesional} asignado.
  \item \textbf{Imagen}: guarda cada fotografía de la úlcera (nombre, fecha, ruta) asociada a un \texttt{Paciente}.
  \item \textbf{Segmentación}: registra cada máscara (método, ruta, fecha) aplicada sobre una \texttt{Imagen}, ya sea manual o automática.
  \item \textbf{PWATScore}: conserva los puntajes PWAT (valor numérico, evaluador, detalles en JSON, fecha), referenciando tanto la \texttt{Imagen} como la \texttt{Segmentación} empleada.
\end{itemize}

Las relaciones principales son:
\begin{itemize}
  \item Un \texttt{User} puede registrar múltiples \texttt{Profesional} y \texttt{Paciente}.
  \item Un \texttt{Profesional} está a cargo de cero o más \texttt{Paciente}.
  \item Cada \texttt{Paciente} puede tener varias \texttt{Imagen}, y cada \texttt{Imagen} puede generar múltiples \texttt{Segmentación} y \texttt{PWATScore}.
  \item Cada \texttt{Segmentación} está vinculada a una única \texttt{Imagen}, y cada \texttt{PWATScore} se apoya en una \texttt{Imagen} y una \texttt{Segmentación}.
\end{itemize}

Este esquema asegura la trazabilidad completa desde el registro de usuarios y pacientes, pasando por la captura y segmentación de imágenes, hasta la evaluación cuantitativa del estadio de la herida mediante puntajes PWAT.

\begin{figure}[H]
  \centering
  \includegraphics[
    width=\textwidth,
    height=0.9\textheight,
    keepaspectratio
  ]{imagenes/modelo_conceptual.png}
  \caption{Diagrama conceptual de entidades y relaciones del sistema.}
  \label{fig:modelo_conceptual}
\end{figure}


\chapter{Diseño}
\label{ch:Impl}    
\section{Diseño Arquitectónico}
\label{sc:DA}

\subsection{Tecnologías utilizadas}
\label{ssc:tech}

El sistema se compone de tres módulos principales, cada uno encargado de una función especifica dentro del flujo de análisis de ulceras.
\subsection{Backend}
\begin{description}
    \item[Función] Exponer un conjunto de servicios REST para gestionar usuarios, pacientes, imágenes y segmentaciones. 
    \item[Persistencia] Un ORM sobre MySQL facilita la definición de las entidades de dominio y sus relaciones 
    \item[Seguridad] Autentificación mediante tokens, junto con un mecanismo de hashing para las credenciales.
    \item[Manejo de archivos] Recepción, almacenamiento y registro de imágenes y mascaras de segmentación.
    \item[Integración con el modelo de ML] Invoca de forma transparente la lógica de análisis y segmentación desarrollada en Python   
\end{description}

\subsection{Frontend}
\begin{description}
\item[Función] Ofrecer una interfaz web para autenticación, visualización de pacientes y gestión de resultados.
\item[Rendimiento y experiencia] Renderizado inicial en servidor y enrutamiento automático para mejorar tiempos de carga y SEO.
\item[Comunicación] Consumo de los servicios del backend, adjuntando el token de autenticación en cada solicitud.
\item[Componentes] Elementos reutilizables para formularios de acceso, panel de imágenes y tablas de datos.
\end{description}

\subsection{Categorizador}
\begin{description}
\item[Función] Ejecución de algoritmos de visión por computador y machine learning para segmentar imágenes y calcular puntajes clínicos.
\item[Tecnologías] Bibliotecas especializadas en procesamiento de imágenes y aprendizaje automático.
\item[Modo de operación]
\begin{itemize}
\item Segmentación de la zona lesionada.
\item Evaluación y asignación de un puntaje en la escala PWAT.
\end{itemize}
\item[Salida] Máscaras segmentadas y valores de puntuación, disponibles para su consulta en el sistema.
\end{description}

\subsection{Visión Unificada}
Este enfoque modular garantiza:
\begin{itemize}
\item Desarrollo y despliegue independientes por componente.
\item Mantenibilidad y escalabilidad a medida que evolucionan los requerimientos.
\item Aprovechamiento de cada tecnología en su área de fortaleza: servicios web, interfaces dinámicas y modelos de inteligencia artificial.
\end{itemize}



\subsection{Flujo de datos / Vista de alto nivel}
\label{ssc:flow}

Para comprender el funcionamiento global de la plataforma, se describira el recorrido que sigue la información desde el momento en que el usuario inicia una acción hasta que obtiene el resultado final. Esta visión de alto nivel permite apreciar de forma clara y ordenada los componentes implicados y la secuencia de procesos que garantizan el correcto manejo de imágenes, metadatos y resultados de segmentación.

En primer lugar, el usuario interactúa mediante la interfaz web, donde sus solicitudes (por ejemplo, carga de imágenes o consulta de resultados previos) son capturadas y enviadas al servidor. A continuación, el módulo de frontend traduce estas peticiones a llamadas a la API REST, incorporando el token de autenticación para validar el acceso. 

El backend recibe la petición, valida los datos y coordina dos tareas principales:  
\begin{enumerate}
  \item \emph{Persistencia}: guarda o recupera los datos relevantes (tipo de usuario: paciente, imágenes, máscaras) en la base de datos.
  \item \emph{Análisis}: cuando se solicita segmentación o cálculo de puntaje, despacha la petición al módulo de categorización, ejecutando el proceso de visión por computador y machine learning en Python.
\end{enumerate}

El módulo de categorizador procesa la imagen, genera la máscara de la lesión y calcula el puntaje clínico, devolviendo estos resultados al backend. Finalmente, el backend almacena la nueva información y responde al frontend con los datos actualizados, que se presentan al usuario mediante componentes interactivos.  

De este modo, el flujo de datos se organiza en cuatro fases —recolección, procesamiento, persistencia y visualización—, garantizando un tránsito de información seguro, escalable y transparente para el investigador o profesional clínico que utilice la plataforma. La Figura \ref{fig:arquitectura-alto-nivel} ilustra estos pasos en un diagrama de alto nivel.  

\begin{figure}[H]
  \centering
  \includegraphics[width=0.8\textwidth,height=0.5\textheight,keepaspectratio]{imagenes/esquemaAlto.png}
  \caption{Esquema de alto nivel del sistema.}
  \label{fig:arquitectura-alto-nivel}
\end{figure}


\subsection{Diagrama de despliegue}
\label{ssc:diagDesp}

\subsubsection{Cliente Web}
Este nodo representa el navegador del usuario, donde se ejecuta la aplicación SPA \footnote{SPA: Single Page Application, es un tipo de aplicación web que se carga una sola vez en el navegador y luego interactúa con el usuario actualizando dinámicamente el contenido de la página sin necesidad de recargarla por completo} construida con Next.js y React. Depende de los artefactos estáticos generados (HTML, CSS y JavaScript) agrupados en un paquete de frontend. La comunicación se realiza sobre HTTP/HTTPS hacia el servidor de la aplicación, enviando solicitudes de carga de imágenes, autentificación y recuperación de datos. Al existir rutas automáticas y rende rizado inicial en servidor, el cliente recibe primero HTML pre-renderizado y luego gestiona de forma reactiva las interacciones con la interfaz

\subsubsection{Servidor de Aplicación}
Agrupa dos paquetes lógicos: el frontend de Next.js y el backend de Express. El paquete de frontend sirve los ficheros estáticos y gestionará el SSR; el de backend expone una API REST para CRUD de usuarios, imágenes y segmentaciones. Internamente, ambos comparten dependencias de Node.js y se despliegan como un mismo contenedor o instancia, garantizando coherencia entre presentación y lógica de negocio. El backend recibe peticiones JSON del cliente, valida tokens y enruta las operaciones según la capa de persistencia o la de análisis.

\subsubsection{Servidor de ML (Categorizador)}
Aislado en una máquina que ejecuta un entorno Python con bibliotecas de visión por computador (OpenCV, PIL, pyradiomics) y aprendizaje automático (scikit-learn, XGBoost, TensorFlow/Keras). Depende de un entorno gestionado por \texttt{Conda} que agrupa todos los módulos necesarios. La interacción con el servidor de aplicación es síncrona: este último envía la imagen a procesar y recibe de vuelta la máscara segmentada y el puntaje PWAT en formato JSON o archivos. De este modo, la lógica de machine learning permanece desacoplada de Node.js.

\subsubsection{Servidor de Base de Datos}
Un servicio MySQL que persiste toda la información clínica y metadatos de imágenes, máscaras y resultados. El backend utiliza un ORM (Sequelize) y el driver mysql2 para comunicarse con este nodo, ejecutando operaciones CRUD. La conexión viaja por la red interna usando el protocolo MySQL; así se separa la capa de datos de las capas de aplicación y análisis, permitiendo escalado independiente y facilidades de respaldo.

\begin{figure}[H]
    \centering
    \includegraphics[width=0.8\textwidth,height=0.5\textheight,keepaspectratio]{imagenes/despliegue.png}
    \caption{Diagrama de Despliegue}
    \label{fig:DD}
\end{figure}


\subsection{Diagrama de componentes}
\label{ssc:diaComp}

A continuacion se presenta el diagrama de componentes que sintetiza la organización modular de la plataforma. En el se identifican las unidades lógicas (Frontend, Backend, Categorizador y sistemas de almacenamiento) y los contratos de servicio (interfaces) que establecen sus puntos de interacción. Esta representación y el flujo de dependencias entre los distintos subsistemas.


\begin{figure}[H]
    \centering
    \includegraphics[width=0.7\linewidth]{imagenes/componentes.png}
    \caption{Diagrama de Componentes}
    \label{fig:componentDiagram}
\end{figure}
    
    
    
\subsection{Diagrama de paquetes}
\label{ssc:pack}
    
A continuación se presenta el diagrama de paquetes que ofrece una visión de alto nivel de la estructura modular del sistema. Cada recuadro agrupa elementos relacionados —páginas, componentes, utilidades y estilos del frontend; configuración, middleware, modelos, controladores y rutas del backend; el script de categorización en Python; y los elementos de persistencia de datos— mostrando cómo se organizan y encapsulan las responsabilidades. Las flechas con estereotipos (\texttt{<<uses>>}, \texttt{<<import>>}, \texttt{<<access>>}) indican las dependencias principales entre paquetes: el frontend consume servicios REST del backend, éste orquesta modelos y controladores y lanza el script de PWAT, y finalmente los módulos acceden a la base de datos y al sistema de archivos para leer o almacenar información. Este esquema facilita entender, de un vistazo, qué agrupa cada paquete y cómo fluye la comunicación entre ellos antes de profundizar en los detalles de implementación.

\begin{figure}[H]
    \centering
    \includegraphics[width=0.7\linewidth]{imagenes/paquetes.png}
    \caption{Diagrama de Paquetes}
    \label{fig:packageDiagram}
\end{figure}
    
\subsection{Diagrama de clases}
\label{ssc:clases}
A continuación se muestra el diagrama de clases que describe el modelo de dominio central de la plataforma. En él aparecen:
Las entidades de gestión de usuarios:
\begin{itemize}
    \item  User
    \item Profesional
    \item Paciente
\end{itemize}
Con sus todos atributos y operaciones principales.

El paquete de contenido clínico
\begin{itemize}
    \item Imagen
    \item Segmentacion
    \item PWATScore
\end{itemize}

Estos modelan la captura, tratamiento y evaluacion de las heridas

Las relaciones y cardinalidades revelan cómo un usuario se vincula a un profesional o paciente, cómo un paciente puede tener múltiples imágenes, y cómo cada imagen da lugar a múltiples segmentaciones y puntuaciones.
Este esquema estático ofrece una visión clara de las piezas independientes del sistema y sus conexiones, sirviendo como referencia para el desarrollo, la integración y la evolución de la plataforma.
\begin{figure}[H]
  \centering
  \adjustbox{%
    max width=\textwidth,       % como mucho el ancho del texto
    max height=1.3\textheight,   % o el 90% de la altura del área imprimible
    keepaspectratio             % sin deformar
  }{%
    \includegraphics{imagenes/clases.png}%
  }
  \caption{Modelo Relacional}
  \label{fig:classesDiagram}
\end{figure}


\section{Diseño de Datos}
\label{sc:DD}

\subsection{Diagrama Entidad Relación}
\label{ssc:ER}
A continuación se presenta el diagrama entidad-relación en notación de Chen \cite{chen1976entity} que sintetiza la estructura lógica de la base de datos. En él se identifican las seis entidades principales (User, Profesional, Paciente, Imagen, Segmentación y PWATScore), sus atributos clave y las relaciones que las vinculan (EsPaciente, EsProfesional, Atiende, Posee, Genera, Evalua y Deriva), con sus cardinalidades asociadas. Este modelo de alto nivel facilita la comprensión del esquema de almacenamiento de datos antes de pasar a detalles de implementación.
\begin{figure}[H]
    \centering
    \includegraphics[width=1\linewidth]{imagenes/ER (1).png}
    \caption{Diagrama Entidad Relacion}
    \label{fig:classesDiagram}
\end{figure}

\subsection{Modelo Relacional}
\label{ssc:Rel}

A continuación se presenta el modelo relacional que refleja la implementación en MySQL de las entidades y relaciones definidas en el esquema ER. Cada rectángulo representa una tabla con sus columnas y restricciones principales (PK, FK, UQ), mientras que las líneas indican las relaciones entre tablas con sus cardinalidades (1:1, 1:N). Este diagrama facilita la lectura de la estructura de la base de datos y sirve de guía para generar las sentencias DDL\footnote{DDL: Data Definition Language o Lenguaje de Definicion de Datos.} correspondientes.
\begin{figure}[H]
  \centering
  \adjustbox{%
    max width=\textwidth,       % como mucho el ancho del texto
    max height=0.70\textheight,   % o el 90% de la altura del área imprimible
    keepaspectratio             % sin deformar
  }{%
    \includegraphics{imagenes/relacional.png}%
  }
  \caption{Modelo Relacional}
  \label{fig:classesDiagram}
\end{figure}



\subsection{Diccionario de Datos}
\label{ssc:DD}
A continuación se incluye el Diccionario de Datos de la base de datos relacional. En él se describen de forma detallada cada una de las tablas —con sus claves primarias, foráneas y restricciones de unicidad— así como los atributos que las componen, su tipo de datos y su significado dentro del sistema. Este artefacto sirve como referencia para el equipo de desarrollo y como documentación formal de la estructura de almacenamiento, facilitando la generación de los scripts DDL y garantizando la consistencia e integridad de la información durante todo el ciclo de vida del proyecto.

\renewcommand{\arraystretch}{1.0}   % Menos altura de fila

\subsection*{Tabla \texttt{User}}
{\footnotesize
\begin{tabularx}{\textwidth}{l l l X}
\hline
\textbf{Atributo} & \textbf{Tipo} & \textbf{Restricción} & \textbf{Descripción} \\\hline
rut               & VARCHAR(20)   & PK                   & RUT del usuario, único e identifica al usuario. \\
nombre            & VARCHAR(100)  & —                    & Nombre completo. \\
correo            & VARCHAR(150)  & UQ                   & Correo electrónico. \\
contrasena\_hash  & VARCHAR(255)  & —                    & Hash de la contraseña. \\
rol               & ENUM          & —                    & Perfil (\texttt{admin}, \texttt{profesional}, \texttt{paciente}). \\
creado\_en        & DATETIME      & —                    & Fecha de creación. \\
fecha\_nacimiento & DATE          & —                    & Fecha de nacimiento. \\\hline
\end{tabularx}
}

\subsection*{Tabla \texttt{Profesional}}
{\footnotesize
\begin{tabularx}{\textwidth}{l l l X}
\hline
\textbf{Atributo} & \textbf{Tipo}       & \textbf{Restricción} & \textbf{Descripción} \\\hline
id                & INT AUTO\_INCREMENT & PK                   & Identificador único. \\
user\_id          & VARCHAR(20)         & FK                   & RUT del usuario. \\
especialidad      & VARCHAR(100)        & —                    & Área de especialidad. \\
fecha\_ingreso    & DATE                & —                    & Fecha de alta. \\\hline
\end{tabularx}
}

\subsection*{Tabla \texttt{Paciente}}
{\footnotesize
\begin{tabularx}{\textwidth}{l l l X}
\hline
\textbf{Atributo} & \textbf{Tipo}              & \textbf{Restricción}  & \textbf{Descripción} \\\hline
id               & INT AUTO\_INCREMENT        & PK                    & Identificador del paciente. \\
sexo             & ENUM                       & —                     & Sexo biológico (\texttt{M}, \texttt{F}). \\
fecha\_ingreso   & DATE                       & —                     & Fecha de registro. \\
comentarios      & TEXT                       & —                     & Notas o comentarios. \\
user\_id         & VARCHAR(20)                & FK                    & RUT de usuario. \\
profesional\_id  & INT                        & FK                    & Profesional a cargo. \\\hline
\end{tabularx}
}

\subsection*{Tabla \texttt{Imagen}}
{\footnotesize
\begin{tabularx}{\textwidth}{l l l X}
\hline
\textbf{Atributo} & \textbf{Tipo}            & \textbf{Restricción} & \textbf{Descripción} \\\hline
id               & INT AUTO\_INCREMENT      & PK                   & Identificador de imagen. \\
fecha\_captura   & DATETIME                 & PK                   & Marca temporal. \\
nombre\_archivo  & VARCHAR(200)             & —                    & Nombre de fichero. \\
ruta\_archivo    & TEXT                     & —                    & Ruta en disco. \\
paciente\_id     & INT                      & FK                   & Paciente asociado. \\\hline
\end{tabularx}
}

\subsection*{Tabla \texttt{Segmentacion}}
{\footnotesize
\begin{tabularx}{\textwidth}{l l l X}
\hline
\textbf{Atributo} & \textbf{Tipo}            & \textbf{Restricción} & \textbf{Descripción} \\\hline
metodo           & ENUM                      & —                    & Tipo (\texttt{manual}, \texttt{automatica}). \\
ruta\_mascara    & TEXT                     & —                    & Ruta de la máscara. \\
fecha\_creacion  & DATETIME                 & —                    & Fecha de creación. \\
imagen\_id       & INT                      & FK PK                  & Imagen asociada. \\\hline
\end{tabularx}
}

\subsection*{Tabla \texttt{PWATScore}}
{\footnotesize
\begin{tabularx}{\textwidth}{l l l X}
\hline
\textbf{Atributo}      & \textbf{Tipo}         & \textbf{Restricción} & \textbf{Descripción} \\\hline
id                    & INT AUTO\_INCREMENT   & PK                   & ID del registro. \\
cat1–cat8             & INT                   & —                    & Categorías clínicas (1–8). \\
fecha\_evaluacion     & DATETIME              & —                    & Fecha de evaluación. \\
observaciones         & TEXT                  & —                    & Comentarios adicionales. \\
imagen\_id            & INT                   & FK                   & Imagen evaluada. \\
segmentacion\_id      & INT                   & FK                   & Segmentación usada. \\\hline
\end{tabularx}
}

\subsection{Diagrama de Flujo de datos}

A continuación se describe el recorrido de la información a través de los distintos módulos de la plataforma:

\begin{enumerate}
  \item \textbf{Interacción de usuario y Frontend}  
    Los componentes de la interfaz (botones, formularios, listados) y las páginas de Next.js reciben las acciones del usuario (por ejemplo, carga de imagen o petición de historial). Estas páginas utilizan el helper de API (\texttt{api.js}) para generar peticiones REST al servidor, adjuntando el token JWT para autorizar cada solicitud.
  
  \item \textbf{Procesamiento en el Backend}  
    El servidor Express recibe cada solicitud y la enruta según su endpoint. A través de las capas de rutas y controladores se valida la entrada, se gestionan credenciales y se aplican reglas de negocio. Los controladores interactúan con los modelos Sequelize para leer o escribir datos en la base de datos MySQL.
  
  \item \textbf{Gestión de archivos y análisis}  
    Cuando corresponde, los controladores también almacenan o recuperan imágenes y máscaras en el sistema de archivos. Para la segmentación automática y el cálculo del PWATScore, el backend invoca de forma asíncrona el script de Python (\texttt{PWAT.py}), pasando la ruta del archivo. El script procesa la imagen, genera las máscaras y devuelve los resultados, que se guardan nuevamente en disco.
  
  \item \textbf{Respuesta al Frontend}  
    Una vez completado el procesamiento —persistencia en la base de datos, almacenamiento de archivos y análisis de ML— el backend construye la respuesta JSON con los datos solicitados (por ejemplo, URLs de las máscaras, valores de puntaje). El frontend recibe estos datos y los muestra en la interfaz, ofreciendo al usuario un feedback inmediato y la posibilidad de seguir navegando o realizar nuevas acciones.
\end{enumerate}

\begin{figure}[H]
    \centering
    \includegraphics[width=1\linewidth]{imagenes/flujo.png}
    \caption{Modelo Relacional}
    \label{fig:classesDiagram}
\end{figure}

\section{Diseño de Interfaz}
\label{sc:DI}

Para prensetar el alcance de la aplicaciom de gestion y revision de imagenes, se empleo un esquema de interfaces de alto nivel, mostrando a su alrededor los cuatro actores principales: \textbf{Administrador}, \textbf{Profesional}, \textbf{Investigador} y \textbf{Paciente}. Cada uno de ellos interactua con el interfaz web (implementada en React/Next.js) para realizar su funcionalidad especifica: el \textbf{Administrador} gestiona usuarios y asignaciones, el \textbf{Investigador} consulta metricas y lanza procesos de rentremamiento, el \textbf{Profesional} lista pacientes y sube o reemplaza imagenes, y el \textbf{Paciente} edita su perfil y revisa sus historiales.


\begin{figure}[H]
    \centering
    \includegraphics[width=1\linewidth]{imagenes/esquema.png}
    \caption{Modelo Relacional}
    \label{fig:classesDiagram}
\end{figure}


%\subsection{Prototipos de Interfaces Gráficas}
%\label{ssc:IGraph}


\section{Diseño de Pruebas}
\label{sc:DP}

Este plan de pruebas se estructura según los principales módulos del sistema: Backend (Node.js/Express), Frontend (Next.js) y Categorizador (Python). Cada componente tiene pruebas asociadas para asegurar su correcto funcionamiento, integración y robustez frente a errores

\subsubsection{Backend}
\textbf{Cobertura de pruebas}
\begin{itemize}
    \item Pruebas Unitarias:
    \begin{itemize}
        \item Cada controlador cuenta con pruebas especificas que verifiquen el comportamiento aislado de sus funciones principales
    \end{itemize}
\begin{itemize}
    \item Se utilizan herramientas como Supertest para simular el entorno de ejecución y mockear dependencias externas
    \item Conexión y operaciones en base de datos de pruebas o uso de mocks de Sequelize
    \item Validación de relaciones entre modelos y respuestas ante inconsistencias
\end{itemize}
\end{itemize}

\subsubsection{Frontend}
\textbf{Cobertura de pruebas}
\begin{itemize}
    \item Pruebas de componentes
    \begin{itemize}
        \item Validación de renderizado correcto y comportamiento de componentes
    \end{itemize}
\begin{itemize}
    \item Simulación de interacciones en formularios
\end{itemize}
\begin{itemize}
    \item Validación de que las peticiones incluyen correctamente los tokens de autenticación
    \item Flujo completo desde login hasta la evaluación de la ulcera
\end{itemize}
\end{itemize}
\subsubsection{Categorizador}
\textbf{Cobertura de pruebas}
\begin{itemize}
    \item Pruebas unitarias
    \begin{itemize}
        \item Escritura y lectura de archivos de imagen y mascaras
        \item Funciones como \texttt{predecir\_mascara}, \texttt{predecir} y \texttt{mask\_predict} se prueban de forma aislada
    \end{itemize}
\end{itemize}

\subsection{Pruebas Unitarias}
\label{ssc:UT}

\begin{enumerate}
    \item \textbf{Backend}
    \begin{enumerate}
        \item \texttt{User.crearUser} debe crear un nuevo usuario con datos validos
        \item \texttt{User.listarUsers} debe retornar todos los usuarios desde la base de datos
        \item \texttt{Pacientes.crearPaciente} debe validar datos requeridos y retornar error si falta
        \item \texttt{Segmentacion.crearSegmentacion} debe recibir la imagen y generar peticion al categorizador
        \item \texttt{PWATScore.calcularPWAT} debe devolver correctamente el puntaje basado en la mascara
    \end{enumerate}
\item \textbf{Frontend}
\begin{enumerate}
    \item Componentes Visuales
    \begin{enumerate}
        \item \texttt{LogiutButton} ejecuta correctamente la función de cierre de sesión
    \end{enumerate}
\item Paginas y Formularios
\begin{enumerate}
    \item \texttt{pages/index.js} redirige al dashborad correcto según rol retornado por la API
    \item Formularios manejan inputs vacíos o inválidos con mensajes adecuados
\end{enumerate}
\item Logica de API
\begin{enumerate}
    \item \texttt{lib/api.js} incluyen correctamente el token en cabeceras
    \item Maneja errores HTTP con feedback al usuario
\end{enumerate}
\end{enumerate}
\item \textbf{Categorizados}
\begin{enumerate}
    \item Funcion de prediccion
    \begin{enumerate}
        \item \texttt{predecir\_mascara} retorna una máscara  valida dado un input conocido
        \item \texttt{predecir} returna una clasificación esperada para inputs de prueba
    \end{enumerate}
\item Errores y Casos Borde
\begin{enumerate}
    \item Responde adecuadamente ante archivos faltantes, rutas invalidas o formatos erróneos
\end{enumerate}
\end{enumerate}
\end{enumerate}

\subsection{Pruebas de Integración}
\label{ssc:IT}

\begin{itemize}
    \item \textbf{Backend}
    \begin{itemize}
        \item Se verifica el funcionamiento conjunto de rutas, middlewares y controladores:
        \begin{itemize}
            \item Pruebas en endpoints como \texttt{POST /users/login}, \texttt{POST /pacientes}, \texttt{POST /segmentaciones}
            \item Confirmación de direccional de errores desde middleware a manejo centralizado
            \item Verificación de acceso a base de datos y correcta respuesta HTTP
        \end{itemize}
    \end{itemize}
\item \textbf{Frontend}
\begin{itemize}
    \item Pruebas sobre la interacción entre componentes, navegación y llamadas a la API
    \begin{itemize}
        \item Confirmación de navegación al dashboard tras autenticación correcta
        \item Validación del comportamiento al recibir errores del backend
        \item Coordinación entre hooks, rutas, y APIs simuladas para pruebas
    \end{itemize}
\end{itemize}
\item \textbf{Categorizador}
\begin{itemize}
    \item Ejecución integrada desde backend Node.js
    \begin{itemize}
        \item Invocación mediante \texttt{child\_process.spawn} o \texttt{conda run}
        \item Recepción y parseo correcto de la respuesta del script \texttt{Python}
        \item Validación de datos temporales generados (mascaras, predicciones)
    \end{itemize}
\end{itemize}
\end{itemize}



\subsection{Pruebas con Usuarios}
\label{ssc:PU}


Este plan evalúa la experiencia de uso del sistema desde la perspectiva de distintos perfiles de usuario: administradores, profesionales de salud y pacientes. Se realizarán pruebas de usabilidad, cumplimiento de flujos esperados y validación de restricciones de acceso.

\subsubsection{Escenarios por Rol}

\begin{itemize}
    \item \textbf{Administrador:}
    \begin{itemize}
        \item Acceso al panel de gestión de usuarios y profesionales.
        \item Creación, edición y eliminación de cuentas.
        \item Acceso completo al historial de segmentaciones.
    \end{itemize}
    
    \item \textbf{Profesional de Salud:}
    \begin{itemize}
        \item Ingreso con credenciales válidas.
        \item Registro de nuevos pacientes.
        \item Carga de imágenes para segmentación.
        \item Visualización de resultados de PWAT.
    \end{itemize}
    
    \item \textbf{Paciente:}
    \begin{itemize}
        \item Acceso restringido a sus propias evaluaciones.
        \item Visualización de resultados previos.
    \end{itemize}
\end{itemize}

\subsubsection{Criterios de Aceptación}

\begin{itemize}
    \item Navegación fluida y sin errores.
    \item Mensajes de retroalimentación claros ante errores o validaciones.
    \item Respeto de los permisos asignados a cada tipo de usuario.
    \item Correcta visualización de resultados y datos asociados.
\end{itemize}

\subsection{Instrumentos}

\begin{itemize}
    \item Pruebas manuales con checklist por escenario.
    \item Grabación de sesiones de prueba para análisis posterior.
    \item Encuesta de satisfacción y usabilidad (\textbf{NASA-TLK} \cite{Hart1988}) aplicada a usuarios reales.
\end{itemize}

\subsection{Definición de métricas de apoyo}\label{ssc:DMA}

Las siguientes métricas permiten cuantificar el desempeño del sistema, evaluar la calidad del software y apoyar la toma de decisiones durante la validación con la escala \textbf{NASA-TLK}:

\begin{itemize}
\item \textbf{Cobertura de pruebas}  \
p. ej. 85\%  \
Porcentaje de funciones, flujos y rutas críticas cubiertas por pruebas automáticas (unitarias, de integración y funcionales).
\item \textbf{Tasa de éxito en pruebas unitarias e integración}  \\
p. ej. 98\%  \\
Número de pruebas que pasan sobre el total de pruebas ejecutadas en cada ciclo.

\item \textbf{Tiempo promedio de respuesta}  \\
p. ej. Login: 250\,ms; Segmentación: 500\,ms; Evaluación: 750\,ms  \\
Promedio de latencia medido en endpoints clave en escenarios de carga realista.

\item \textbf{Tasa de errores encontrados por usuarios}  \\
p. ej. 2 incidencias por ciclo de pruebas  \\
Número de defectos o fallos reportados durante las sesiones de pruebas con usuarios (ambiente de validación).

\item \textbf{Puntuación NASA-TLK}  \\
p. Escala de 1 a 7 por dimensión  \\
Evaluación estandarizada de usabilidad centrada en:  \\
   \begin{itemize}
       \item \textbf{Carga Mental}
       \item \textbf{Carga Física}
       \item \textbf{Carga Temporal}
       \item \textbf{Desempeño}
       \item \textbf{Esfuerzo}
       \item \textbf{Satisfacción}
       \item \textbf{Curva de Aprendizaje}
   \end{itemize}
Se recogen las valoraciones de los usuarios tras completar tareas esenciales.
\end{itemize}

\noindent\textbf{Uso y frecuencia de análisis:}
\begin{itemize}
\item Se calculan tras cada iteración de pruebas de usabilidad para identificar áreas críticas de mejora.
\item Se comparan tendencias temporales para validar estabilidad y evolución de la experiencia clínica.

\end{itemize}


\chapter{Recursos}
\label{ch:rec}
\input{recursos}


%\chapter{Pruebas}
%\label{ch:Pruebas}
%%Para evaluar el sistema desarrollado se deben presentar y discutir al menos las siguientes pruebas:

\section{Pruebas Unitarias}
\label{sc:UT}

\section{Pruebas de Integración}
\label{sc:IT}

\section{Pruebas de Sistema}
\label{ssc:PS}

\section{Pruebas con Usuarios}
\label{sc:PU}

\section{Pruebas de Aceptación}
\label{sc:PA}

    
%\chapter{Implantación}
%\label{ch:Impla}
%\section{Implantación}
\label{sc:Impl}

\subsection{Requerimientos }

\subsubsection{Requerimientos de Mínimos}

\subsubsection{Requerimientos de Recomendados}

\subsection{Preparación de Ambiente} 
%Debe considerar por ejemplo cómo configurar BD, como instalar y levantar servicios. 


\subsection{Documentación}

\subsection{Manual de Usuario}

\subsection{Documentación de desarrollo} 
%ejemplo HTML con documentación o API generada para desarrollo. 


%\chapter{Conclusiones}
%\label{ch:Concl}
%\section{Conclusiones}
\label{sc:Concl}

\begin{itemize}
    \item Breve introducción al trabajo realizado.
    \item Descripción de cumplimiento de objetivos iniciales (para entrega final).
    \item Presentación de dificultades durante el trabajo.
    \item Presentación de limitaciones y proyecciones del trabajo o trabajos futuros.
\end{itemize}

%En las conclusiones no se presenta nada nuevo. 

%Use tiempo pretérito para el resumen de los resultados, y tiempo presente para los "outcomes".

%Debe ser presentado en orden de importancia (más a menos). 

%Cada conclusión debe ser soportada por evidencia. Resultados inesperados deben ser presentados y explicados. 


%\appendix
%\include{appendix1}
%\include{appendix2}
%\include{appendix3}

\bibliographystyle{IEEEtran}
\bibliography{referencias}

\end{document} 