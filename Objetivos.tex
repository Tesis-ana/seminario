\section{Objetivos}

\subsection{Objetivo General}
\label{sc:OG}
Desarrollar e implementar, en un periodo máximo de 12 meses, una plataforma web y una aplicación móvil que integren una herramienta automatizada para apoyar la evaluación objetiva del estadio de heridas ulcerosas en pacientes con pie diabético, mediante la estimación automática del valor de la escala PWAT (Photographic Wound Assessment Tool), alcanzando al menos un 90% de concordancia con la valoración de especialistas.

\subsection{Objetivos Específicos}
\label{ssc:OE}

\begin{itemize}
    \item Seleccionar y evaluar, durante los primeros cuatro meses, modelos de aprendizaje automático para la predicción automática de categorías de la escala PWAT, buscando una correlación mínima de 0.85 con la evaluación de expertos y garantizando la precisión y la utilidad clínica en la clasificación de heridas ulcerosas
    \item Integrar antes del mes ocho un procedimiento de segmentación de imágenes que permita obtener la máscara de la herida, sea generada automáticamente o trazada por el usuario, alcanzando un IoU superior a 0.70 para delimitar con precisión la zona afectada y mejorar la consistencia en la clasificación.
    \item Validar la solución propuesta en una prueba piloto de al menos diez casos clínicos antes del mes doce, comparándola con enfoques tradicionales y obteniendo una satisfacción de usuarios mayor al 80%.
\end{itemize}

